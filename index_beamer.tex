% Options for packages loaded elsewhere
% Options for packages loaded elsewhere
\PassOptionsToPackage{unicode}{hyperref}
\PassOptionsToPackage{hyphens}{url}
%
\documentclass[
  ignorenonframetext,
]{beamer}
\newif\ifbibliography
\usepackage{pgfpages}
\setbeamertemplate{caption}[numbered]
\setbeamertemplate{caption label separator}{: }
\setbeamercolor{caption name}{fg=normal text.fg}
\beamertemplatenavigationsymbolsempty
% remove section numbering
\setbeamertemplate{part page}{
  \centering
  \begin{beamercolorbox}[sep=16pt,center]{part title}
    \usebeamerfont{part title}\insertpart\par
  \end{beamercolorbox}
}
\setbeamertemplate{section page}{
  \centering
  \begin{beamercolorbox}[sep=12pt,center]{section title}
    \usebeamerfont{section title}\insertsection\par
  \end{beamercolorbox}
}
\setbeamertemplate{subsection page}{
  \centering
  \begin{beamercolorbox}[sep=8pt,center]{subsection title}
    \usebeamerfont{subsection title}\insertsubsection\par
  \end{beamercolorbox}
}
% Prevent slide breaks in the middle of a paragraph
\widowpenalties 1 10000
\raggedbottom
\AtBeginPart{
  \frame{\partpage}
}
\AtBeginSection{
  \ifbibliography
  \else
    \frame{\sectionpage}
  \fi
}
\AtBeginSubsection{
  \frame{\subsectionpage}
}
\usepackage{iftex}
\ifPDFTeX
  \usepackage[T1]{fontenc}
  \usepackage[utf8]{inputenc}
  \usepackage{textcomp} % provide euro and other symbols
\else % if luatex or xetex
  \usepackage{unicode-math} % this also loads fontspec
  \defaultfontfeatures{Scale=MatchLowercase}
  \defaultfontfeatures[\rmfamily]{Ligatures=TeX,Scale=1}
\fi
\usepackage{lmodern}

\ifPDFTeX\else
  % xetex/luatex font selection
\fi
% Use upquote if available, for straight quotes in verbatim environments
\IfFileExists{upquote.sty}{\usepackage{upquote}}{}
\IfFileExists{microtype.sty}{% use microtype if available
  \usepackage[]{microtype}
  \UseMicrotypeSet[protrusion]{basicmath} % disable protrusion for tt fonts
}{}
\makeatletter
\@ifundefined{KOMAClassName}{% if non-KOMA class
  \IfFileExists{parskip.sty}{%
    \usepackage{parskip}
  }{% else
    \setlength{\parindent}{0pt}
    \setlength{\parskip}{6pt plus 2pt minus 1pt}}
}{% if KOMA class
  \KOMAoptions{parskip=half}}
\makeatother

\usepackage{color}
\usepackage{fancyvrb}
\newcommand{\VerbBar}{|}
\newcommand{\VERB}{\Verb[commandchars=\\\{\}]}
\DefineVerbatimEnvironment{Highlighting}{Verbatim}{commandchars=\\\{\}}
% Add ',fontsize=\small' for more characters per line
\newenvironment{Shaded}{}{}
\newcommand{\AlertTok}[1]{\textcolor[rgb]{1.00,0.00,0.00}{\textbf{#1}}}
\newcommand{\AnnotationTok}[1]{\textcolor[rgb]{0.38,0.63,0.69}{\textbf{\textit{#1}}}}
\newcommand{\AttributeTok}[1]{\textcolor[rgb]{0.49,0.56,0.16}{#1}}
\newcommand{\BaseNTok}[1]{\textcolor[rgb]{0.25,0.63,0.44}{#1}}
\newcommand{\BuiltInTok}[1]{\textcolor[rgb]{0.00,0.50,0.00}{#1}}
\newcommand{\CharTok}[1]{\textcolor[rgb]{0.25,0.44,0.63}{#1}}
\newcommand{\CommentTok}[1]{\textcolor[rgb]{0.38,0.63,0.69}{\textit{#1}}}
\newcommand{\CommentVarTok}[1]{\textcolor[rgb]{0.38,0.63,0.69}{\textbf{\textit{#1}}}}
\newcommand{\ConstantTok}[1]{\textcolor[rgb]{0.53,0.00,0.00}{#1}}
\newcommand{\ControlFlowTok}[1]{\textcolor[rgb]{0.00,0.44,0.13}{\textbf{#1}}}
\newcommand{\DataTypeTok}[1]{\textcolor[rgb]{0.56,0.13,0.00}{#1}}
\newcommand{\DecValTok}[1]{\textcolor[rgb]{0.25,0.63,0.44}{#1}}
\newcommand{\DocumentationTok}[1]{\textcolor[rgb]{0.73,0.13,0.13}{\textit{#1}}}
\newcommand{\ErrorTok}[1]{\textcolor[rgb]{1.00,0.00,0.00}{\textbf{#1}}}
\newcommand{\ExtensionTok}[1]{#1}
\newcommand{\FloatTok}[1]{\textcolor[rgb]{0.25,0.63,0.44}{#1}}
\newcommand{\FunctionTok}[1]{\textcolor[rgb]{0.02,0.16,0.49}{#1}}
\newcommand{\ImportTok}[1]{\textcolor[rgb]{0.00,0.50,0.00}{\textbf{#1}}}
\newcommand{\InformationTok}[1]{\textcolor[rgb]{0.38,0.63,0.69}{\textbf{\textit{#1}}}}
\newcommand{\KeywordTok}[1]{\textcolor[rgb]{0.00,0.44,0.13}{\textbf{#1}}}
\newcommand{\NormalTok}[1]{#1}
\newcommand{\OperatorTok}[1]{\textcolor[rgb]{0.40,0.40,0.40}{#1}}
\newcommand{\OtherTok}[1]{\textcolor[rgb]{0.00,0.44,0.13}{#1}}
\newcommand{\PreprocessorTok}[1]{\textcolor[rgb]{0.74,0.48,0.00}{#1}}
\newcommand{\RegionMarkerTok}[1]{#1}
\newcommand{\SpecialCharTok}[1]{\textcolor[rgb]{0.25,0.44,0.63}{#1}}
\newcommand{\SpecialStringTok}[1]{\textcolor[rgb]{0.73,0.40,0.53}{#1}}
\newcommand{\StringTok}[1]{\textcolor[rgb]{0.25,0.44,0.63}{#1}}
\newcommand{\VariableTok}[1]{\textcolor[rgb]{0.10,0.09,0.49}{#1}}
\newcommand{\VerbatimStringTok}[1]{\textcolor[rgb]{0.25,0.44,0.63}{#1}}
\newcommand{\WarningTok}[1]{\textcolor[rgb]{0.38,0.63,0.69}{\textbf{\textit{#1}}}}

\usepackage{longtable,booktabs,array}
\usepackage{calc} % for calculating minipage widths
\usepackage{caption}
% Make caption package work with longtable
\makeatletter
\def\fnum@table{\tablename~\thetable}
\makeatother
\usepackage{graphicx}
\makeatletter
\newsavebox\pandoc@box
\newcommand*\pandocbounded[1]{% scales image to fit in text height/width
  \sbox\pandoc@box{#1}%
  \Gscale@div\@tempa{\textheight}{\dimexpr\ht\pandoc@box+\dp\pandoc@box\relax}%
  \Gscale@div\@tempb{\linewidth}{\wd\pandoc@box}%
  \ifdim\@tempb\p@<\@tempa\p@\let\@tempa\@tempb\fi% select the smaller of both
  \ifdim\@tempa\p@<\p@\scalebox{\@tempa}{\usebox\pandoc@box}%
  \else\usebox{\pandoc@box}%
  \fi%
}
% Set default figure placement to htbp
\def\fps@figure{htbp}
\makeatother


% definitions for citeproc citations
\NewDocumentCommand\citeproctext{}{}
\NewDocumentCommand\citeproc{mm}{%
  \begingroup\def\citeproctext{#2}\cite{#1}\endgroup}
\makeatletter
 % allow citations to break across lines
 \let\@cite@ofmt\@firstofone
 % avoid brackets around text for \cite:
 \def\@biblabel#1{}
 \def\@cite#1#2{{#1\if@tempswa , #2\fi}}
\makeatother
\newlength{\cslhangindent}
\setlength{\cslhangindent}{1.5em}
\newlength{\csllabelwidth}
\setlength{\csllabelwidth}{3em}
\newenvironment{CSLReferences}[2] % #1 hanging-indent, #2 entry-spacing
 {\begin{list}{}{%
  \setlength{\itemindent}{0pt}
  \setlength{\leftmargin}{0pt}
  \setlength{\parsep}{0pt}
  % turn on hanging indent if param 1 is 1
  \ifodd #1
   \setlength{\leftmargin}{\cslhangindent}
   \setlength{\itemindent}{-1\cslhangindent}
  \fi
  % set entry spacing
  \setlength{\itemsep}{#2\baselineskip}}}
 {\end{list}}
\usepackage{calc}
\newcommand{\CSLBlock}[1]{\hfill\break\parbox[t]{\linewidth}{\strut\ignorespaces#1\strut}}
\newcommand{\CSLLeftMargin}[1]{\parbox[t]{\csllabelwidth}{\strut#1\strut}}
\newcommand{\CSLRightInline}[1]{\parbox[t]{\linewidth - \csllabelwidth}{\strut#1\strut}}
\newcommand{\CSLIndent}[1]{\hspace{\cslhangindent}#1}



\setlength{\emergencystretch}{3em} % prevent overfull lines

\providecommand{\tightlist}{%
  \setlength{\itemsep}{0pt}\setlength{\parskip}{0pt}}



 


\usepackage{booktabs}
\usepackage{caption}
\usepackage{longtable}
\usepackage{colortbl}
\usepackage{array}
\usepackage{anyfontsize}
\usepackage{multirow}
% ======================= CONFIGURAÇÕES DE ESTILO ===========================
\definecolor{UFPELBlue}{RGB}{20, 49, 129}

% Cores e fonte
\setbeamercolor{headline}{bg=UFPELBlue}
\setbeamercolor{frametitle}{bg=white, fg=UFPELBlue}
\setbeamercolor{title}{fg=UFPELBlue}
\usefonttheme{default}

% Pacotes necessários
\usepackage[absolute,overlay]{textpos}
\usepackage{setspace}
\usepackage{array}
\usepackage{etoolbox}
\usepackage{graphicx}

% ======================= TEMPLATE DA PÁGINA DE TÍTULO ===========================

\defbeamertemplate*{title page}{custom-title-page}{
  \vbox to \textheight{
    % Logo IFM
    \begin{textblock*}{\textwidth}(0.2cm, 0.4cm)
      \includegraphics[height=1cm]{logo_ifm.png}
    \end{textblock*}

    % Instituto
    \begin{textblock*}{\textwidth}(1cm, 0.5cm)
      {\usebeamerfont{institute}\tiny
      \setlength{\baselineskip}{1.5pt}
      \centering
      \renewcommand{\arraystretch}{0.2}
      \begin{tabular}{c}
        \insertinstitute
      \end{tabular}
      \par}
    \end{textblock*}

    \vfill
 % Define comando para substituir \and por \\ para autores no título
\makeatletter
\def\authorswithnewline{\def\and{\\}\insertauthor}
\makeatother


    \begin{center}
      % Título
      {\usebeamerfont{title}\usebeamercolor[fg]{title}\textbf{\inserttitle}\par}
      \vskip0.8cm
      
     

      % Autores um embaixo do outro (FORÇA)
  {\usebeamerfont{author}\usebeamercolor[fg]{author}
  \tiny
    % LISTE OS AUTORES AQUI, UM POR UM, COM \\ NO FINAL DE CADA UM
    ANA RITA DE ASSUMPÇÃO MAZZINI \\
    GISELDA MARIA PEREIRA \\
    POLLYANE VIEIRA DA SILVA \\
    ISADORA MOREIRA DA LUZ REAL \\
    ANA LUIZA BARBOZA MERLIN \\
    FERNANDO NEUGEBAUER REHBEIN DA CUNHA PENEDO \\
    LUCAS DE AZEVEDO DE SOUZA
    \par
  }
      \vskip0.4cm

      % Data
      {\usebeamerfont{date}\insertdate\par}
    \end{center}

    \vfill

    % Logos inferiores
    \begin{textblock*}{\textwidth}(11.4cm, 8cm)
      \includegraphics[height=1.2cm]{logo_ufpel_seal.png}
    \end{textblock*}

    \begin{textblock*}{\textwidth}(0.2cm, 8.4cm)
      \includegraphics[height=0.8cm]{RStudio.png}
    \end{textblock*}
  }
}
% ======================= OUTROS AJUSTES ===========================

% Frame title centralizado
\setbeamertemplate{frametitle}{%
  \begin{center}\textbf{\insertframetitle}\end{center}
}

% Barra azul superior em todos os frames
\setbeamertemplate{headline}{%
  \leavevmode
  \hbox{%
    \begin{beamercolorbox}[wd=\paperwidth,ht=2.5ex,dp=1.5ex]{headline}
    \end{beamercolorbox}%
  }
  % Logo R no canto superior direito
  \begin{textblock*}{2cm}(0.88\paperwidth,0.4cm)
    \includegraphics[height=1cm]{logo_r.png}
  \end{textblock*}
}

% Barra azul inferior em todos os frames
\setbeamertemplate{footline}{%
  \leavevmode
  \hbox{%
    \begin{beamercolorbox}[wd=\paperwidth,ht=2.5ex,dp=1.5ex]{headline}
    \end{beamercolorbox}%
  }
}

\AtBeginEnvironment{quote}{\small}
\makeatletter
\@ifpackageloaded{caption}{}{\usepackage{caption}}
\AtBeginDocument{%
\ifdefined\contentsname
  \renewcommand*\contentsname{Table of contents}
\else
  \newcommand\contentsname{Table of contents}
\fi
\ifdefined\listfigurename
  \renewcommand*\listfigurename{List of Figures}
\else
  \newcommand\listfigurename{List of Figures}
\fi
\ifdefined\listtablename
  \renewcommand*\listtablename{List of Tables}
\else
  \newcommand\listtablename{List of Tables}
\fi
\ifdefined\figurename
  \renewcommand*\figurename{Figure}
\else
  \newcommand\figurename{Figure}
\fi
\ifdefined\tablename
  \renewcommand*\tablename{Table}
\else
  \newcommand\tablename{Table}
\fi
}
\@ifpackageloaded{float}{}{\usepackage{float}}
\floatstyle{ruled}
\@ifundefined{c@chapter}{\newfloat{codelisting}{h}{lop}}{\newfloat{codelisting}{h}{lop}[chapter]}
\floatname{codelisting}{Listing}
\newcommand*\listoflistings{\listof{codelisting}{List of Listings}}
\makeatother
\makeatletter
\makeatother
\makeatletter
\@ifpackageloaded{caption}{}{\usepackage{caption}}
\@ifpackageloaded{subcaption}{}{\usepackage{subcaption}}
\makeatother

\usepackage{bookmark}
\IfFileExists{xurl.sty}{\usepackage{xurl}}{} % add URL line breaks if available
\urlstyle{same}
\hypersetup{
  pdftitle={Explorando o pacote ggplot2 do R},
  hidelinks,
  pdfcreator={LaTeX via pandoc}}


\title{Explorando o pacote ggplot2 do R}
\author{}
\date{}

\begin{document}
\frame{\titlepage}


\begin{frame}[fragile]{Introdução}
\phantomsection\label{introduuxe7uxe3o}
O \texttt{ggplot2} é um pacote de código aberto para a visualização
gráfica de dados para a linguagem de programação R. Foi criada por
Hadley Wickham em 2005 (Wickham 2016), sendo uma implementação do livro
\texttt{Grammar\ Graphics} de Leland Wilkison também lançado em 2005
(Wilkinson 2011).

Ele aborda que visualização gráfica dos dados pode ser divida em
componentes semânticos, como escalas e camadas.

\begin{center}
\includegraphics[width=0.2\linewidth,height=\textheight,keepaspectratio]{Ggplot2_hex_logo.svg.png}
\end{center}
\end{frame}

\begin{frame}[fragile]{Por que usar o ggplot2?}
\phantomsection\label{por-que-usar-o-ggplot2}
\begin{enumerate}
\item
  Alta costumização gráfica.
\item
  Alta diversidade de modelos de gráficos.
\item
  Integração com outros pacotes do tidyverse, como por exemplo
  \texttt{dplyr} (Wickham et al. 2023), \texttt{forcats} (Wickham 2023)
  e o \texttt{plotly} (Sievert 2020).
\item
  Criação de gráficos a partir de camadas, podendo sobrepor diferentes
  gráficos.
\end{enumerate}
\end{frame}

\begin{frame}[fragile]{Como instalar o ggplot2?}
\phantomsection\label{como-instalar-o-ggplot2}
\begin{columns}[T]
\begin{column}{0.4\linewidth}
\textbf{Instalando pacotes}

\begin{Shaded}
\begin{Highlighting}[]
\CommentTok{\#instalando pacote ggplot2}
\FunctionTok{install.packages}\NormalTok{(}\StringTok{"ggplot2"}\NormalTok{)}

\CommentTok{\#instalando dplyr, forcats }
\CommentTok{\# e patchwork}
\FunctionTok{install.packages}\NormalTok{(}\StringTok{"dplyr"}\NormalTok{)}
\FunctionTok{install.packages}\NormalTok{(}\StringTok{"forcats"}\NormalTok{)}
\FunctionTok{install.packages}\NormalTok{(}\StringTok{"patchwork"}\NormalTok{)}
\end{Highlighting}
\end{Shaded}
\end{column}

\begin{column}{0.4\linewidth}
\textbf{Carregando pacotes}

\begin{Shaded}
\begin{Highlighting}[]
\CommentTok{\#Carregando o pacote ggplot2}
\FunctionTok{library}\NormalTok{(ggplot2)}

\CommentTok{\#Carregando dplyr, forcats}
\CommentTok{\#e patchwork}
\FunctionTok{library}\NormalTok{(dplyr)}
\FunctionTok{library}\NormalTok{(forcats)}
\FunctionTok{library}\NormalTok{(patchwork)}
\end{Highlighting}
\end{Shaded}
\end{column}
\end{columns}
\end{frame}

\begin{frame}[fragile]{Banco de dados \emph{iris}}
\phantomsection\label{banco-de-dados-iris}
Para essa oficina será utilizado bancos de dados \textbf{iris}.

\textbf{iris} - é referente tamanho de pételas e sépalas de 3 espécies
do gênero \emph{Iris} do trabalho de Fisher em 1936 (\emph{Iris
setosa},\emph{Iris versicolor} e \emph{Iris virginica})

\begin{Shaded}
\begin{Highlighting}[]
\FunctionTok{data}\NormalTok{(iris)}
\end{Highlighting}
\end{Shaded}

\begin{table}
\fontsize{7.5pt}{9.0pt}\selectfont
\begin{tabular*}{\linewidth}{@{\extracolsep{\fill}}rrrrc}
\toprule
Sepal.Length & Sepal.Width & Petal.Length & Petal.Width & Species \\ 
\midrule\addlinespace[2.5pt]
5.1 & 3.5 & 1.4 & 0.2 & setosa \\ 
4.9 & 3.0 & 1.4 & 0.2 & setosa \\ 
4.7 & 3.2 & 1.3 & 0.2 & setosa \\ 
4.6 & 3.1 & 1.5 & 0.2 & setosa \\ 
\bottomrule
\end{tabular*}
\end{table}
\end{frame}

\begin{frame}[fragile]{Box-plot}
\phantomsection\label{box-plot}
\begin{Shaded}
\begin{Highlighting}[]
\NormalTok{iris}\SpecialCharTok{\%\textgreater{}\%}\FunctionTok{ggplot}\NormalTok{(}\FunctionTok{aes}\NormalTok{(}\AttributeTok{x=}\NormalTok{Species, }\AttributeTok{y=}\NormalTok{Petal.Length))}\SpecialCharTok{+}
  \FunctionTok{geom\_boxplot}\NormalTok{()}
\end{Highlighting}
\end{Shaded}

\pandocbounded{\includegraphics[keepaspectratio]{index_beamer_files/figure-beamer/Box-plot simples1-1.pdf}}
\end{frame}

\begin{frame}[fragile]{Gráfico violino}
\phantomsection\label{gruxe1fico-violino}
\begin{Shaded}
\begin{Highlighting}[]
\FunctionTok{ggplot}\NormalTok{(iris, }\FunctionTok{aes}\NormalTok{(}\AttributeTok{x=}\NormalTok{Species,}\AttributeTok{y=}\NormalTok{Sepal.Width, }\AttributeTok{fill=}\NormalTok{Species))}\SpecialCharTok{+}
  \FunctionTok{geom\_violin}\NormalTok{()}
\end{Highlighting}
\end{Shaded}

\pandocbounded{\includegraphics[keepaspectratio]{index_beamer_files/figure-beamer/violino-1.pdf}}
\end{frame}

\begin{frame}[fragile]{Histograma}
\phantomsection\label{histograma}
\begin{Shaded}
\begin{Highlighting}[]
\FunctionTok{ggplot}\NormalTok{(iris,}\FunctionTok{aes}\NormalTok{(}\AttributeTok{x=}\NormalTok{Sepal.Width))}\SpecialCharTok{+}
  \FunctionTok{geom\_histogram}\NormalTok{(}\AttributeTok{bins=}\DecValTok{10}\NormalTok{, }\AttributeTok{color=}\StringTok{"black"}\NormalTok{,}
                 \AttributeTok{fill=}\StringTok{"white"}\NormalTok{)}\SpecialCharTok{+}
  \FunctionTok{labs}\NormalTok{(}\AttributeTok{y=}\StringTok{"Frequência"}\NormalTok{, }\AttributeTok{x=}\StringTok{"Largura de Sépala"}\NormalTok{)}
\end{Highlighting}
\end{Shaded}

\pandocbounded{\includegraphics[keepaspectratio]{index_beamer_files/figure-beamer/histograma-1.pdf}}
\end{frame}

\begin{frame}[fragile]
\begin{Shaded}
\begin{Highlighting}[]
\FunctionTok{ggplot}\NormalTok{(iris,}\FunctionTok{aes}\NormalTok{(}\AttributeTok{x=}\NormalTok{Sepal.Width))}\SpecialCharTok{+}
  \FunctionTok{geom\_histogram}\NormalTok{(}\AttributeTok{bins=}\DecValTok{11}\NormalTok{, }\AttributeTok{color=}\StringTok{"black"}\NormalTok{,}
                 \AttributeTok{fill=}\StringTok{"white"}\NormalTok{)}\SpecialCharTok{+}
  \FunctionTok{labs}\NormalTok{(}\AttributeTok{y=}\StringTok{"Frequência"}\NormalTok{, }\AttributeTok{x=}\StringTok{"Largura de Sépala"}\NormalTok{)}\SpecialCharTok{+}
  \FunctionTok{scale\_x\_continuous}\NormalTok{(}\AttributeTok{n.breaks =} \DecValTok{11}\NormalTok{)}
\end{Highlighting}
\end{Shaded}

\pandocbounded{\includegraphics[keepaspectratio]{index_beamer_files/figure-beamer/unnamed-chunk-2-1.pdf}}
\end{frame}

\begin{frame}[fragile]{Polígono de frequências}
\phantomsection\label{poluxedgono-de-frequuxeancias}
\begin{Shaded}
\begin{Highlighting}[]
\FunctionTok{ggplot}\NormalTok{(iris,}\FunctionTok{aes}\NormalTok{(}\AttributeTok{x=}\NormalTok{Sepal.Width))}\SpecialCharTok{+}
  \FunctionTok{geom\_freqpoly}\NormalTok{(}\AttributeTok{bins=}\DecValTok{11}\NormalTok{, }\AttributeTok{color=}\StringTok{"black"}\NormalTok{)}\SpecialCharTok{+}
  \FunctionTok{labs}\NormalTok{(}\AttributeTok{y=}\StringTok{"Frequência"}\NormalTok{, }\AttributeTok{x=}\StringTok{"Largura de Sépala"}\NormalTok{)}\SpecialCharTok{+}
  \FunctionTok{scale\_x\_continuous}\NormalTok{(}\AttributeTok{n.breaks =} \DecValTok{11}\NormalTok{)}
\end{Highlighting}
\end{Shaded}

\pandocbounded{\includegraphics[keepaspectratio]{index_beamer_files/figure-beamer/histograma1-1.pdf}}
\end{frame}

\begin{frame}[fragile]
\begin{Shaded}
\begin{Highlighting}[]
\FunctionTok{ggplot}\NormalTok{(iris,}\FunctionTok{aes}\NormalTok{(}\AttributeTok{x=}\NormalTok{Sepal.Width))}\SpecialCharTok{+}
  \FunctionTok{labs}\NormalTok{(}\AttributeTok{y=}\StringTok{"Frequência"}\NormalTok{, }\AttributeTok{x=}\StringTok{"Largura de Sépala"}\NormalTok{)}\SpecialCharTok{+}
  \FunctionTok{scale\_x\_continuous}\NormalTok{(}\AttributeTok{n.breaks =} \DecValTok{11}\NormalTok{)}\SpecialCharTok{+}
  \FunctionTok{geom\_histogram}\NormalTok{(}\AttributeTok{bins=}\DecValTok{11}\NormalTok{, }\AttributeTok{color=}\StringTok{"black"}\NormalTok{,}
                 \AttributeTok{fill=}\StringTok{"white"}\NormalTok{)}\SpecialCharTok{+}
  \FunctionTok{geom\_freqpoly}\NormalTok{(}\AttributeTok{bins=}\DecValTok{11}\NormalTok{, }\AttributeTok{color=}\StringTok{"blue"}\NormalTok{)}
\end{Highlighting}
\end{Shaded}

\pandocbounded{\includegraphics[keepaspectratio]{index_beamer_files/figure-beamer/unnamed-chunk-3-1.pdf}}
\end{frame}

\begin{frame}[fragile]
\begin{Shaded}
\begin{Highlighting}[]
\FunctionTok{ggplot}\NormalTok{(iris,}\FunctionTok{aes}\NormalTok{(}\AttributeTok{x=}\NormalTok{Sepal.Width))}\SpecialCharTok{+}
  \FunctionTok{labs}\NormalTok{(}\AttributeTok{y=}\StringTok{"Frequência"}\NormalTok{, }\AttributeTok{x=}\StringTok{"Largura de Sépala"}\NormalTok{)}\SpecialCharTok{+}
  \FunctionTok{scale\_x\_continuous}\NormalTok{(}\AttributeTok{n.breaks =} \DecValTok{11}\NormalTok{)}\SpecialCharTok{+}
  \FunctionTok{geom\_histogram}\NormalTok{(}\AttributeTok{bins=}\DecValTok{11}\NormalTok{, }\AttributeTok{color=}\StringTok{"black"}\NormalTok{,}
                 \AttributeTok{fill=}\StringTok{"white"}\NormalTok{)}\SpecialCharTok{+}
  \FunctionTok{geom\_freqpoly}\NormalTok{(}\AttributeTok{bins=}\DecValTok{11}\NormalTok{, }\AttributeTok{color=}\StringTok{"blue"}\NormalTok{)}\SpecialCharTok{+}
  \FunctionTok{facet\_grid}\NormalTok{(}\SpecialCharTok{\textasciitilde{}}\NormalTok{Species)}
\end{Highlighting}
\end{Shaded}

\pandocbounded{\includegraphics[keepaspectratio]{index_beamer_files/figure-beamer/unnamed-chunk-4-1.pdf}}
\end{frame}

\begin{frame}[fragile]
\begin{Shaded}
\begin{Highlighting}[]
\FunctionTok{ggplot}\NormalTok{(iris,}\FunctionTok{aes}\NormalTok{(}\AttributeTok{x=}\NormalTok{Sepal.Width))}\SpecialCharTok{+}
  \FunctionTok{labs}\NormalTok{(}\AttributeTok{y=}\StringTok{"Frequência"}\NormalTok{, }\AttributeTok{x=}\StringTok{"Largura de Sépala"}\NormalTok{)}\SpecialCharTok{+}
  \FunctionTok{scale\_x\_continuous}\NormalTok{(}\AttributeTok{n.breaks =} \DecValTok{11}\NormalTok{)}\SpecialCharTok{+}
  \FunctionTok{geom\_histogram}\NormalTok{(}\AttributeTok{bins=}\DecValTok{11}\NormalTok{, }\AttributeTok{color=}\StringTok{"black"}\NormalTok{,}
                 \AttributeTok{fill=}\StringTok{"white"}\NormalTok{)}\SpecialCharTok{+}
  \FunctionTok{geom\_freqpoly}\NormalTok{(}\AttributeTok{bins=}\DecValTok{11}\NormalTok{, }\AttributeTok{color=}\StringTok{"blue"}\NormalTok{)}\SpecialCharTok{+}
  \FunctionTok{facet\_grid}\NormalTok{(Species}\SpecialCharTok{\textasciitilde{}}\NormalTok{.)}
\end{Highlighting}
\end{Shaded}

\pandocbounded{\includegraphics[keepaspectratio]{index_beamer_files/figure-beamer/unnamed-chunk-5-1.pdf}}
\end{frame}

\begin{frame}[fragile]{Gráfico de densidade}
\phantomsection\label{gruxe1fico-de-densidade}
\begin{Shaded}
\begin{Highlighting}[]
\FunctionTok{ggplot}\NormalTok{(iris,}\FunctionTok{aes}\NormalTok{(}\AttributeTok{x=}\NormalTok{Sepal.Width))}\SpecialCharTok{+}
  \FunctionTok{geom\_density}\NormalTok{(}\AttributeTok{color=}\StringTok{"black"}\NormalTok{, }\AttributeTok{fill=}\StringTok{"white"}\NormalTok{)}\SpecialCharTok{+}
  \FunctionTok{labs}\NormalTok{(}\AttributeTok{y=}\StringTok{"Frequência"}\NormalTok{, }\AttributeTok{x=}\StringTok{"Largura de Sépala"}\NormalTok{)}
\end{Highlighting}
\end{Shaded}

\pandocbounded{\includegraphics[keepaspectratio]{index_beamer_files/figure-beamer/unnamed-chunk-6-1.pdf}}
\end{frame}

\begin{frame}[fragile]{Gráfico de barras de frequência}
\phantomsection\label{gruxe1fico-de-barras-de-frequuxeancia}
\begin{Shaded}
\begin{Highlighting}[]
\NormalTok{iris}\SpecialCharTok{\%\textgreater{}\%}\FunctionTok{ggplot}\NormalTok{(}\FunctionTok{aes}\NormalTok{(}\AttributeTok{x=}\NormalTok{Species))}\SpecialCharTok{+}
  \FunctionTok{geom\_bar}\NormalTok{()}
\end{Highlighting}
\end{Shaded}

\pandocbounded{\includegraphics[keepaspectratio]{index_beamer_files/figure-beamer/Frequência-1.pdf}}
\end{frame}

\begin{frame}[fragile]
\begin{Shaded}
\begin{Highlighting}[]
\NormalTok{iris}\SpecialCharTok{\%\textgreater{}\%}\FunctionTok{group\_by}\NormalTok{(Species)}\SpecialCharTok{\%\textgreater{}\%}
  \FunctionTok{summarise}\NormalTok{(}\AttributeTok{count=}\FunctionTok{n}\NormalTok{())}\SpecialCharTok{\%\textgreater{}\%}
  \FunctionTok{ggplot}\NormalTok{(}\FunctionTok{aes}\NormalTok{(}\AttributeTok{x=}\NormalTok{Species, }\AttributeTok{fill=}\NormalTok{Species, }\AttributeTok{y=}\NormalTok{count))}\SpecialCharTok{+}
  \FunctionTok{geom\_col}\NormalTok{(}\AttributeTok{color=}\StringTok{"black"}\NormalTok{)}
\end{Highlighting}
\end{Shaded}

\pandocbounded{\includegraphics[keepaspectratio]{index_beamer_files/figure-beamer/unnamed-chunk-7-1.pdf}}
\end{frame}

\begin{frame}[fragile]{Gráfico de setores}
\phantomsection\label{gruxe1fico-de-setores}
\begin{Shaded}
\begin{Highlighting}[]
\NormalTok{iris}\SpecialCharTok{\%\textgreater{}\%}\FunctionTok{group\_by}\NormalTok{(Species)}\SpecialCharTok{\%\textgreater{}\%}
  \FunctionTok{summarise}\NormalTok{(}\AttributeTok{count=}\FunctionTok{n}\NormalTok{()}\SpecialCharTok{/}\DecValTok{150}\SpecialCharTok{*}\DecValTok{100}\NormalTok{)}\SpecialCharTok{\%\textgreater{}\%}
  \FunctionTok{ggplot}\NormalTok{(}\FunctionTok{aes}\NormalTok{(}\AttributeTok{x=}\StringTok{" "}\NormalTok{, }\AttributeTok{fill=}\NormalTok{Species, }\AttributeTok{y=}\NormalTok{count))}\SpecialCharTok{+}
  \FunctionTok{geom\_col}\NormalTok{(}\AttributeTok{color=}\StringTok{"black"}\NormalTok{)}\SpecialCharTok{+}
  \FunctionTok{coord\_polar}\NormalTok{(}\AttributeTok{theta=}\StringTok{"y"}\NormalTok{)}\SpecialCharTok{+}
  \FunctionTok{theme\_void}\NormalTok{()}
\end{Highlighting}
\end{Shaded}

\pandocbounded{\includegraphics[keepaspectratio]{index_beamer_files/figure-beamer/unnamed-chunk-8-1.pdf}}
\end{frame}

\begin{frame}[fragile]
\begin{Shaded}
\begin{Highlighting}[]
\NormalTok{iris}\SpecialCharTok{\%\textgreater{}\%}\FunctionTok{group\_by}\NormalTok{(Species)}\SpecialCharTok{\%\textgreater{}\%}
  \FunctionTok{summarise}\NormalTok{(}\AttributeTok{count=}\FunctionTok{round}\NormalTok{(}\FunctionTok{n}\NormalTok{()}\SpecialCharTok{/}\DecValTok{150}\SpecialCharTok{*}\DecValTok{100}\NormalTok{, }\DecValTok{2}\NormalTok{))}\SpecialCharTok{\%\textgreater{}\%}
  \FunctionTok{ggplot}\NormalTok{(}\FunctionTok{aes}\NormalTok{(}\AttributeTok{x=}\StringTok{" "}\NormalTok{, }\AttributeTok{fill=}\NormalTok{Species, }\AttributeTok{y=}\NormalTok{count))}\SpecialCharTok{+}
  \FunctionTok{geom\_col}\NormalTok{(}\AttributeTok{color=}\StringTok{"black"}\NormalTok{)}\SpecialCharTok{+}\FunctionTok{coord\_polar}\NormalTok{(}\AttributeTok{theta=}\StringTok{"y"}\NormalTok{)}\SpecialCharTok{+} 
  \FunctionTok{geom\_label}\NormalTok{(}\FunctionTok{aes}\NormalTok{(}\AttributeTok{label =} \FunctionTok{paste0}\NormalTok{(count, }\StringTok{"\%"}\NormalTok{)),}
             \AttributeTok{position =} \FunctionTok{position\_stack}\NormalTok{(}\AttributeTok{vjust =} \FloatTok{0.5}\NormalTok{),}
             \AttributeTok{show.legend =} \ConstantTok{FALSE}\NormalTok{, }\AttributeTok{size=}\DecValTok{3}\NormalTok{)}\SpecialCharTok{+}
  \FunctionTok{theme\_void}\NormalTok{()}
\end{Highlighting}
\end{Shaded}

\pandocbounded{\includegraphics[keepaspectratio]{index_beamer_files/figure-beamer/unnamed-chunk-9-1.pdf}}
\end{frame}

\begin{frame}[fragile]{Diagrama de pontos}
\phantomsection\label{diagrama-de-pontos}
\begin{Shaded}
\begin{Highlighting}[]
\FunctionTok{ggplot}\NormalTok{(iris,}\FunctionTok{aes}\NormalTok{(}\AttributeTok{x=}\NormalTok{Sepal.Length, }\AttributeTok{y=}\NormalTok{Sepal.Width))}\SpecialCharTok{+}
  \FunctionTok{geom\_point}\NormalTok{()}
\end{Highlighting}
\end{Shaded}

\pandocbounded{\includegraphics[keepaspectratio]{index_beamer_files/figure-beamer/unnamed-chunk-10-1.pdf}}
\end{frame}

\begin{frame}[fragile]
\begin{Shaded}
\begin{Highlighting}[]
\FunctionTok{ggplot}\NormalTok{(iris,}\FunctionTok{aes}\NormalTok{(}\AttributeTok{x=}\NormalTok{Sepal.Length, }\AttributeTok{y=}\NormalTok{Sepal.Width,}
                \AttributeTok{color=}\NormalTok{Species, }\AttributeTok{shape=}\NormalTok{Species))}\SpecialCharTok{+}
  \FunctionTok{geom\_point}\NormalTok{()}
\end{Highlighting}
\end{Shaded}

\pandocbounded{\includegraphics[keepaspectratio]{index_beamer_files/figure-beamer/unnamed-chunk-11-1.pdf}}
\end{frame}

\begin{frame}[fragile]
\begin{Shaded}
\begin{Highlighting}[]
\FunctionTok{ggplot}\NormalTok{(iris,}\FunctionTok{aes}\NormalTok{(}\AttributeTok{x=}\NormalTok{Sepal.Length, }\AttributeTok{y=}\NormalTok{Sepal.Width,}
                \AttributeTok{color=}\NormalTok{Species, }\AttributeTok{shape=}\NormalTok{Species))}\SpecialCharTok{+}
  \FunctionTok{geom\_point}\NormalTok{()}\SpecialCharTok{+}
  \FunctionTok{geom\_smooth}\NormalTok{(}\AttributeTok{se=}\ConstantTok{FALSE}\NormalTok{, }\AttributeTok{method=}\StringTok{"lm"}\NormalTok{)}
\end{Highlighting}
\end{Shaded}

\pandocbounded{\includegraphics[keepaspectratio]{index_beamer_files/figure-beamer/unnamed-chunk-12-1.pdf}}
\end{frame}

\begin{frame}[fragile]
\begin{Shaded}
\begin{Highlighting}[]
\FunctionTok{ggplot}\NormalTok{(iris,}\FunctionTok{aes}\NormalTok{(}\AttributeTok{x=}\NormalTok{Sepal.Length, }\AttributeTok{y=}\NormalTok{Sepal.Width, }\AttributeTok{color=}\NormalTok{Species,}
                \AttributeTok{shape=}\NormalTok{Species))}\SpecialCharTok{+}
  \FunctionTok{geom\_point}\NormalTok{()}\SpecialCharTok{+}
  \FunctionTok{geom\_smooth}\NormalTok{(}\AttributeTok{se=}\ConstantTok{FALSE}\NormalTok{, }\AttributeTok{method=}\StringTok{"lm"}\NormalTok{)}\SpecialCharTok{+}
  \FunctionTok{coord\_flip}\NormalTok{()}
\end{Highlighting}
\end{Shaded}

\pandocbounded{\includegraphics[keepaspectratio]{index_beamer_files/figure-beamer/unnamed-chunk-13-1.pdf}}
\end{frame}

\begin{frame}[fragile]{Gráfico de barras (média e desvio)}
\phantomsection\label{gruxe1fico-de-barras-muxe9dia-e-desvio}
\begin{Shaded}
\begin{Highlighting}[]
\NormalTok{iris}\SpecialCharTok{\%\textgreater{}\%}\FunctionTok{group\_by}\NormalTok{(Species)}\SpecialCharTok{\%\textgreater{}\%}
  \FunctionTok{summarise}\NormalTok{(}\AttributeTok{mean=}\FunctionTok{mean}\NormalTok{(Sepal.Length),}
            \AttributeTok{sd=}\FunctionTok{sd}\NormalTok{(Sepal.Length),}
            \AttributeTok{se=}\FunctionTok{sd}\NormalTok{(Sepal.Length)}\SpecialCharTok{/}\FunctionTok{sqrt}\NormalTok{(}\FunctionTok{length}\NormalTok{(Sepal.Length)))}\SpecialCharTok{\%\textgreater{}\%}
  \FunctionTok{ggplot}\NormalTok{(}\FunctionTok{aes}\NormalTok{(}\AttributeTok{x=}\NormalTok{Species, }\AttributeTok{y=}\NormalTok{mean))}\SpecialCharTok{+}
  \FunctionTok{geom\_col}\NormalTok{()}\SpecialCharTok{+}
  \FunctionTok{geom\_errorbar}\NormalTok{(}\FunctionTok{aes}\NormalTok{(}\AttributeTok{ymin=}\NormalTok{mean}\SpecialCharTok{{-}}\NormalTok{sd,}\AttributeTok{ymax=}\NormalTok{mean}\SpecialCharTok{+}\NormalTok{sd), }\AttributeTok{width=}\FloatTok{0.5}\NormalTok{)}\SpecialCharTok{+}
  \FunctionTok{labs}\NormalTok{(}\AttributeTok{y=}\StringTok{"Comprimento da Sepala"}\NormalTok{, }\AttributeTok{x=}\StringTok{"Espécies"}\NormalTok{)}\SpecialCharTok{+}
  \FunctionTok{theme\_bw}\NormalTok{()}\SpecialCharTok{+}
  \FunctionTok{scale\_y\_continuous}\NormalTok{(}\AttributeTok{limits=}\FunctionTok{c}\NormalTok{(}\DecValTok{0}\NormalTok{,}\DecValTok{10}\NormalTok{))}
\end{Highlighting}
\end{Shaded}
\end{frame}

\begin{frame}
\pandocbounded{\includegraphics[keepaspectratio]{index_beamer_files/figure-beamer/grafico de barras-1.pdf}}
\end{frame}

\begin{frame}[fragile]
\begin{Shaded}
\begin{Highlighting}[]
\NormalTok{iris}\SpecialCharTok{\%\textgreater{}\%}\FunctionTok{group\_by}\NormalTok{(Species)}\SpecialCharTok{\%\textgreater{}\%}
  \FunctionTok{summarise}\NormalTok{(}\AttributeTok{mean=}\FunctionTok{mean}\NormalTok{(Sepal.Length),}
            \AttributeTok{sd=}\FunctionTok{sd}\NormalTok{(Sepal.Length),}
            \AttributeTok{se=}\FunctionTok{sd}\NormalTok{(Sepal.Length)}\SpecialCharTok{/}\FunctionTok{sqrt}\NormalTok{(}\FunctionTok{length}\NormalTok{(Sepal.Length)))}\SpecialCharTok{\%\textgreater{}\%}
  \FunctionTok{ggplot}\NormalTok{(}\FunctionTok{aes}\NormalTok{(}\AttributeTok{x=}\NormalTok{Species, }\AttributeTok{y=}\NormalTok{mean))}\SpecialCharTok{+}
  \FunctionTok{geom\_col}\NormalTok{()}\SpecialCharTok{+}
  \FunctionTok{geom\_linerange}\NormalTok{(}\FunctionTok{aes}\NormalTok{(}\AttributeTok{ymin=}\NormalTok{mean}\SpecialCharTok{{-}}\NormalTok{sd,}\AttributeTok{ymax=}\NormalTok{mean}\SpecialCharTok{+}\NormalTok{sd))}\SpecialCharTok{+}
  \FunctionTok{labs}\NormalTok{(}\AttributeTok{y=}\StringTok{"Comprimento da Sepala"}\NormalTok{, }\AttributeTok{x=}\StringTok{"Espécies"}\NormalTok{)}\SpecialCharTok{+}
  \FunctionTok{theme\_bw}\NormalTok{()}\SpecialCharTok{+}
  \FunctionTok{scale\_y\_continuous}\NormalTok{(}\AttributeTok{limits=}\FunctionTok{c}\NormalTok{(}\DecValTok{0}\NormalTok{,}\DecValTok{10}\NormalTok{))}
\end{Highlighting}
\end{Shaded}
\end{frame}

\begin{frame}
\pandocbounded{\includegraphics[keepaspectratio]{index_beamer_files/figure-beamer/unnamed-chunk-15-1.pdf}}
\end{frame}

\begin{frame}[fragile]
\begin{Shaded}
\begin{Highlighting}[]
\NormalTok{iris}\SpecialCharTok{\%\textgreater{}\%}\FunctionTok{group\_by}\NormalTok{(Species)}\SpecialCharTok{\%\textgreater{}\%}
  \FunctionTok{summarise}\NormalTok{(}\AttributeTok{mean=}\FunctionTok{mean}\NormalTok{(Sepal.Length), }\AttributeTok{sd=}\FunctionTok{sd}\NormalTok{(Sepal.Length),}\AttributeTok{se=}\FunctionTok{sd}\NormalTok{(Sepal.Length)}\SpecialCharTok{/}\FunctionTok{sqrt}\NormalTok{(}\FunctionTok{length}\NormalTok{(Sepal.Length)))}\SpecialCharTok{\%\textgreater{}\%}
  \FunctionTok{ggplot}\NormalTok{(}\FunctionTok{aes}\NormalTok{(}\AttributeTok{x=}\NormalTok{Species, }\AttributeTok{y=}\NormalTok{mean))}\SpecialCharTok{+}
  \FunctionTok{geom\_col}\NormalTok{()}\SpecialCharTok{+}
  \FunctionTok{geom\_pointrange}\NormalTok{(}\FunctionTok{aes}\NormalTok{(}\AttributeTok{ymin=}\NormalTok{mean}\SpecialCharTok{{-}}\NormalTok{sd,}\AttributeTok{ymax=}\NormalTok{mean}\SpecialCharTok{+}\NormalTok{sd))}\SpecialCharTok{+}
  \FunctionTok{labs}\NormalTok{(}\AttributeTok{y=}\StringTok{"Comprimento da Sepala"}\NormalTok{, }\AttributeTok{x=}\StringTok{"Espécies"}\NormalTok{)}\SpecialCharTok{+}
  \FunctionTok{theme\_bw}\NormalTok{()}\SpecialCharTok{+}
  \FunctionTok{scale\_y\_continuous}\NormalTok{(}\AttributeTok{limits=}\FunctionTok{c}\NormalTok{(}\DecValTok{0}\NormalTok{,}\DecValTok{10}\NormalTok{))}
\end{Highlighting}
\end{Shaded}
\end{frame}

\begin{frame}
\pandocbounded{\includegraphics[keepaspectratio]{index_beamer_files/figure-beamer/unnamed-chunk-17-1.pdf}}
\end{frame}

\begin{frame}[fragile]
\begin{Shaded}
\begin{Highlighting}[]
\NormalTok{iris}\SpecialCharTok{\%\textgreater{}\%}\FunctionTok{group\_by}\NormalTok{(Species)}\SpecialCharTok{\%\textgreater{}\%}
  \FunctionTok{summarise}\NormalTok{(}\AttributeTok{mean=}\FunctionTok{mean}\NormalTok{(Sepal.Length), }\AttributeTok{sd=}\FunctionTok{sd}\NormalTok{(Sepal.Length),}\AttributeTok{se=}\FunctionTok{sd}\NormalTok{(Sepal.Length)}\SpecialCharTok{/}\FunctionTok{sqrt}\NormalTok{(}\FunctionTok{length}\NormalTok{(Sepal.Length)))}\SpecialCharTok{\%\textgreater{}\%}
  \FunctionTok{ggplot}\NormalTok{(}\FunctionTok{aes}\NormalTok{(}\AttributeTok{x=}\NormalTok{Species, }\AttributeTok{y=}\NormalTok{mean))}\SpecialCharTok{+}
  \FunctionTok{geom\_linerange}\NormalTok{(}\FunctionTok{aes}\NormalTok{(}\AttributeTok{ymin=}\NormalTok{mean}\SpecialCharTok{{-}}\NormalTok{sd,}\AttributeTok{ymax=}\NormalTok{mean}\SpecialCharTok{+}\NormalTok{sd))}\SpecialCharTok{+}
  \FunctionTok{labs}\NormalTok{(}\AttributeTok{y=}\StringTok{"Comprimento da Sepala"}\NormalTok{, }\AttributeTok{x=}\StringTok{"Espécies"}\NormalTok{)}\SpecialCharTok{+}
  \FunctionTok{theme\_bw}\NormalTok{()}\SpecialCharTok{+}
  \FunctionTok{scale\_y\_continuous}\NormalTok{(}\AttributeTok{limits=}\FunctionTok{c}\NormalTok{(}\DecValTok{0}\NormalTok{,}\DecValTok{10}\NormalTok{))}
\end{Highlighting}
\end{Shaded}
\end{frame}

\begin{frame}
\pandocbounded{\includegraphics[keepaspectratio]{index_beamer_files/figure-beamer/unnamed-chunk-19-1.pdf}}
\end{frame}

\section{Alterando escalas, cores, fontes e
temas}\label{alterando-escalas-cores-fontes-e-temas}

\begin{frame}[fragile]{Ajustando escalas no ggplot}
\phantomsection\label{ajustando-escalas-no-ggplot}
\begin{Shaded}
\begin{Highlighting}[]
\NormalTok{iris}\SpecialCharTok{\%\textgreater{}\%}\FunctionTok{group\_by}\NormalTok{(Species)}\SpecialCharTok{\%\textgreater{}\%}
  \FunctionTok{summarise}\NormalTok{(}\AttributeTok{mean=}\FunctionTok{mean}\NormalTok{(Sepal.Length), }
            \AttributeTok{sd=}\FunctionTok{sd}\NormalTok{(Sepal.Length),}
            \AttributeTok{se=}\FunctionTok{sd}\NormalTok{(Sepal.Length)}\SpecialCharTok{/}\FunctionTok{sqrt}\NormalTok{(}\FunctionTok{length}\NormalTok{(Sepal.Length)))}\SpecialCharTok{\%\textgreater{}\%}
  \FunctionTok{ggplot}\NormalTok{(}\FunctionTok{aes}\NormalTok{(}\AttributeTok{x=}\NormalTok{Species, }\AttributeTok{y=}\NormalTok{mean))}\SpecialCharTok{+}
  \FunctionTok{geom\_col}\NormalTok{()}\SpecialCharTok{+}
  \FunctionTok{geom\_errorbar}\NormalTok{(}\FunctionTok{aes}\NormalTok{(}\AttributeTok{ymin=}\NormalTok{mean}\SpecialCharTok{{-}}\NormalTok{sd,}\AttributeTok{ymax=}\NormalTok{mean}\SpecialCharTok{+}\NormalTok{sd), }\AttributeTok{width=}\FloatTok{0.5}\NormalTok{)}\SpecialCharTok{+}
  \FunctionTok{labs}\NormalTok{(}\AttributeTok{y=}\StringTok{"Comprimento da Sepala"}\NormalTok{, }\AttributeTok{x=}\StringTok{"Espécies"}\NormalTok{)}\SpecialCharTok{+}
  \FunctionTok{theme\_bw}\NormalTok{()}\SpecialCharTok{+}
  \FunctionTok{scale\_y\_continuous}\NormalTok{(}\AttributeTok{limits=}\FunctionTok{c}\NormalTok{(}\DecValTok{0}\NormalTok{,}\DecValTok{10}\NormalTok{))}
\end{Highlighting}
\end{Shaded}
\end{frame}

\begin{frame}
\pandocbounded{\includegraphics[keepaspectratio]{index_beamer_files/figure-beamer/escala1-1.pdf}}
\end{frame}

\begin{frame}[fragile]
\begin{Shaded}
\begin{Highlighting}[]
\CommentTok{\#exemplo com a escala maior}
\NormalTok{iris}\SpecialCharTok{\%\textgreater{}\%}\FunctionTok{group\_by}\NormalTok{(Species)}\SpecialCharTok{\%\textgreater{}\%}
  \FunctionTok{summarise}\NormalTok{(}\AttributeTok{mean=}\FunctionTok{mean}\NormalTok{(Sepal.Length),}
            \AttributeTok{sd=}\FunctionTok{sd}\NormalTok{(Sepal.Length),}
            \AttributeTok{se=}\FunctionTok{sd}\NormalTok{(Sepal.Length)}\SpecialCharTok{/}\FunctionTok{sqrt}\NormalTok{(}\FunctionTok{length}\NormalTok{(Sepal.Length)))}\SpecialCharTok{\%\textgreater{}\%}
  \FunctionTok{ggplot}\NormalTok{(}\FunctionTok{aes}\NormalTok{(}\AttributeTok{x=}\NormalTok{Species, }\AttributeTok{y=}\NormalTok{mean))}\SpecialCharTok{+}
  \FunctionTok{geom\_col}\NormalTok{()}\SpecialCharTok{+}
  \FunctionTok{geom\_errorbar}\NormalTok{(}\FunctionTok{aes}\NormalTok{(}\AttributeTok{ymin=}\NormalTok{mean}\SpecialCharTok{{-}}\NormalTok{sd,}\AttributeTok{ymax=}\NormalTok{mean}\SpecialCharTok{+}\NormalTok{sd), }\AttributeTok{width=}\FloatTok{0.5}\NormalTok{)}\SpecialCharTok{+}
  \FunctionTok{labs}\NormalTok{(}\AttributeTok{y=}\StringTok{"Comprimento da Sepala"}\NormalTok{, }\AttributeTok{x=}\StringTok{"Espécies"}\NormalTok{)}\SpecialCharTok{+}
  \FunctionTok{theme\_bw}\NormalTok{()}\SpecialCharTok{+}
  \FunctionTok{scale\_y\_continuous}\NormalTok{(}\AttributeTok{limits=}\FunctionTok{c}\NormalTok{(}\DecValTok{0}\NormalTok{,}\DecValTok{20}\NormalTok{))}
\end{Highlighting}
\end{Shaded}
\end{frame}

\begin{frame}
\pandocbounded{\includegraphics[keepaspectratio]{index_beamer_files/figure-beamer/unnamed-chunk-21-1.pdf}}
\end{frame}

\begin{frame}[fragile]{Ordenando variáveis ordinais no ggplot}
\phantomsection\label{ordenando-variuxe1veis-ordinais-no-ggplot}
\begin{Shaded}
\begin{Highlighting}[]
\NormalTok{Escolaridade}\OtherTok{\textless{}{-}}\FunctionTok{c}\NormalTok{(}\FunctionTok{rep}\NormalTok{(}\StringTok{"Graduação"}\NormalTok{, }\DecValTok{42}\NormalTok{),}
                \FunctionTok{rep}\NormalTok{(}\StringTok{"Médio"}\NormalTok{, }\DecValTok{30}\NormalTok{),}
                \FunctionTok{rep}\NormalTok{(}\StringTok{"Fundamental"}\NormalTok{, }\DecValTok{20}\NormalTok{))}

\NormalTok{Escolaridade}\OtherTok{\textless{}{-}}\FunctionTok{as.data.frame}\NormalTok{(Escolaridade)}

\NormalTok{Escolaridade}\SpecialCharTok{\%\textgreater{}\%}\FunctionTok{ggplot}\NormalTok{(}\FunctionTok{aes}\NormalTok{(}\AttributeTok{x=}\NormalTok{Escolaridade))}\SpecialCharTok{+}
  \FunctionTok{geom\_bar}\NormalTok{()}\SpecialCharTok{+}
  \FunctionTok{labs}\NormalTok{(}\AttributeTok{y=}\StringTok{"Frequência"}\NormalTok{, }\AttributeTok{x=}\StringTok{"Escolaridade"}\NormalTok{)}
\end{Highlighting}
\end{Shaded}
\end{frame}

\begin{frame}
\pandocbounded{\includegraphics[keepaspectratio]{index_beamer_files/figure-beamer/unnamed-chunk-23-1.pdf}}
\end{frame}

\begin{frame}[fragile]
\begin{Shaded}
\begin{Highlighting}[]
\NormalTok{Escolaridade}\SpecialCharTok{\%\textgreater{}\%}\FunctionTok{mutate}\NormalTok{(}\AttributeTok{Escolaridade=}\FunctionTok{fct\_relevel}\NormalTok{(Escolaridade,}
                                               \StringTok{"Fundamental"}\NormalTok{,}
                                               \StringTok{"Médio"}\NormalTok{,}
                                               \StringTok{"Graduação"}\NormalTok{))}\SpecialCharTok{\%\textgreater{}\%}
  \FunctionTok{ggplot}\NormalTok{(}\FunctionTok{aes}\NormalTok{(}\AttributeTok{x=}\NormalTok{Escolaridade))}\SpecialCharTok{+}
  \FunctionTok{geom\_bar}\NormalTok{()}\SpecialCharTok{+}
  \FunctionTok{labs}\NormalTok{(}\AttributeTok{y=}\StringTok{"Frequência"}\NormalTok{, }\AttributeTok{x=}\StringTok{"Escolaridade"}\NormalTok{)}
\end{Highlighting}
\end{Shaded}
\end{frame}

\begin{frame}
\pandocbounded{\includegraphics[keepaspectratio]{index_beamer_files/figure-beamer/unnamed-chunk-25-1.pdf}}
\end{frame}

\begin{frame}[fragile]{Mudando cores de preenchimento no ggplot}
\phantomsection\label{mudando-cores-de-preenchimento-no-ggplot}
\begin{Shaded}
\begin{Highlighting}[]
\NormalTok{iris}\SpecialCharTok{\%\textgreater{}\%}\FunctionTok{ggplot}\NormalTok{(}\FunctionTok{aes}\NormalTok{(}\AttributeTok{x=}\NormalTok{Species, }\AttributeTok{y=}\NormalTok{Petal.Length, }\AttributeTok{fill=}\NormalTok{Species))}\SpecialCharTok{+}
  \FunctionTok{geom\_boxplot}\NormalTok{()}
\end{Highlighting}
\end{Shaded}

\pandocbounded{\includegraphics[keepaspectratio]{index_beamer_files/figure-beamer/Box-plot1-1.pdf}}
\end{frame}

\begin{frame}[fragile]
\begin{Shaded}
\begin{Highlighting}[]
\NormalTok{iris}\SpecialCharTok{\%\textgreater{}\%}\FunctionTok{ggplot}\NormalTok{(}\FunctionTok{aes}\NormalTok{(}\AttributeTok{x=}\NormalTok{Species, }\AttributeTok{y=}\NormalTok{Petal.Length))}\SpecialCharTok{+}
  \FunctionTok{geom\_boxplot}\NormalTok{(}\AttributeTok{fill=}\FunctionTok{c}\NormalTok{(}\StringTok{"lightpink"}\NormalTok{,}\StringTok{"lightgreen"}\NormalTok{,}\StringTok{"lightblue"}\NormalTok{))}
\end{Highlighting}
\end{Shaded}

\pandocbounded{\includegraphics[keepaspectratio]{index_beamer_files/figure-beamer/Box-plot2-1.pdf}}
\end{frame}

\begin{frame}[fragile]
\begin{Shaded}
\begin{Highlighting}[]
\NormalTok{iris}\SpecialCharTok{\%\textgreater{}\%}\FunctionTok{ggplot}\NormalTok{(}\FunctionTok{aes}\NormalTok{(}\AttributeTok{x=}\NormalTok{Species, }\AttributeTok{y=}\NormalTok{Petal.Length, }\AttributeTok{fill=}\NormalTok{Species))}\SpecialCharTok{+}
  \FunctionTok{geom\_boxplot}\NormalTok{()}\SpecialCharTok{+}
  \FunctionTok{scale\_fill\_manual}\NormalTok{(}\AttributeTok{values=}\FunctionTok{c}\NormalTok{(}\StringTok{"\#704c41"}\NormalTok{,}\StringTok{"\#41704f"}\NormalTok{,}\StringTok{"\#584170"}\NormalTok{))}
\end{Highlighting}
\end{Shaded}

\pandocbounded{\includegraphics[keepaspectratio]{index_beamer_files/figure-beamer/Box-plot3-1.pdf}}
\end{frame}

\begin{frame}[fragile]{Mudando cores de contorno no ggplot}
\phantomsection\label{mudando-cores-de-contorno-no-ggplot}
\begin{Shaded}
\begin{Highlighting}[]
\NormalTok{iris}\SpecialCharTok{\%\textgreater{}\%}\FunctionTok{ggplot}\NormalTok{(}\FunctionTok{aes}\NormalTok{(}\AttributeTok{x=}\NormalTok{Species, }\AttributeTok{y=}\NormalTok{Petal.Length, }\AttributeTok{fill=}\NormalTok{Species))}\SpecialCharTok{+}
  \FunctionTok{geom\_boxplot}\NormalTok{(}\AttributeTok{fill=}\FunctionTok{c}\NormalTok{(}\StringTok{"lightblue"}\NormalTok{,}\StringTok{"lightgreen"}\NormalTok{,}\StringTok{"lightpink"}\NormalTok{),}
               \AttributeTok{color=}\StringTok{"brown"}\NormalTok{)}
\end{Highlighting}
\end{Shaded}

\pandocbounded{\includegraphics[keepaspectratio]{index_beamer_files/figure-beamer/Box-plot4-1.pdf}}
\end{frame}

\begin{frame}{Alterando elementos textuais no ggplot}
\phantomsection\label{alterando-elementos-textuais-no-ggplot}
Os nomes dos eixos são alterados pela função labs, onde você indica qual
elemento gráfico você quer renomear. Lembre-se: o nome que você quer
renomear tem que estar entre aspas \textbf{'' ``}.

\begin{itemize}
\tightlist
\item
  \textbf{y} para alterar o título do eixo y.
\item
  \textbf{x} para alterar o título do eixo x.
\item
  \textbf{title} para alterar o título ou acrescentar um título.
\item
  \textbf{subtitle} para alterar o subtítulo ou acrescentar um
  subtítulo.
\item
  \textbf{fill} para alterar o título da legenda referente ao fator
  colocado no fill.
\item
  \textbf{color} para alterar o título da legenda referente ao fator
  colocado no color.
\item
  \textbf{shape} para alterar o título da legenda referente ao fator
  colocado no shape.
\item
  \textbf{size} para alterar o título da legenda referente ao fator
  colocado no size.
\end{itemize}
\end{frame}

\begin{frame}[fragile]
\begin{Shaded}
\begin{Highlighting}[]
\NormalTok{iris}\SpecialCharTok{\%\textgreater{}\%}\FunctionTok{ggplot}\NormalTok{(}\FunctionTok{aes}\NormalTok{(}\AttributeTok{x=}\NormalTok{Species, }\AttributeTok{y=}\NormalTok{Petal.Length, }\AttributeTok{fill=}\NormalTok{Species))}\SpecialCharTok{+}
  \FunctionTok{geom\_boxplot}\NormalTok{(}\AttributeTok{fill=}\FunctionTok{c}\NormalTok{(}\StringTok{"lightblue"}\NormalTok{,}\StringTok{"lightgreen"}\NormalTok{,}\StringTok{"lightpink"}\NormalTok{),}
               \AttributeTok{color=}\StringTok{"brown"}\NormalTok{)}\SpecialCharTok{+}
  \FunctionTok{labs}\NormalTok{(}\AttributeTok{y=}\StringTok{"Comprimento de pétala"}\NormalTok{,}
       \AttributeTok{x=}\StringTok{"Espécies"}\NormalTok{,}
       \AttributeTok{title=}\StringTok{"Comparação de comprimento de pétalas"}\NormalTok{,}
       \AttributeTok{subtitle =} \StringTok{"Banco de dados iris"}\NormalTok{)}
\end{Highlighting}
\end{Shaded}

\pandocbounded{\includegraphics[keepaspectratio]{index_beamer_files/figure-beamer/Box-plot eixos-1.pdf}}
\end{frame}

\begin{frame}[fragile]{Alterando a fonte}
\phantomsection\label{alterando-a-fonte}
\begin{Shaded}
\begin{Highlighting}[]
\CommentTok{\# Instalando o pacote extrafont}
\FunctionTok{install.packages}\NormalTok{(}\StringTok{"extrafont"}\NormalTok{)}

\CommentTok{\#Carregando o pacote extrafont}
\FunctionTok{library}\NormalTok{(extrafont)}

\CommentTok{\#Carregando as fontes presentes no computador}
\FunctionTok{loadfonts}\NormalTok{(}\AttributeTok{device=}\StringTok{"all"}\NormalTok{)}
\end{Highlighting}
\end{Shaded}
\end{frame}

\begin{frame}[fragile]
Aqui alteramos as fontes através do comando \texttt{theme()} este
comando altera elementos temáticos do gráfico, como por exemplo fontes,
tamanhos, cor de fundo, entre outros. Neste exemplo colocamons o
argumento \texttt{text\ =\ element\_text()}. Dentro dele vai alguns
argumentos:

\begin{itemize}
\tightlist
\item
  \textbf{face} é para definir se a fonte estará em itálico
  (\texttt{"italic"}), negrito (\texttt{"bold"}) ou ambos
  (\texttt{"italic.bold"})
\item
  \textbf{family} é para definir se o tipo de fonte. Esse argumento pode
  ter variações de acordo com sistema operacional do computador. Em
  sistema windows pode-se utilizar \texttt{"TT\ Times\ New\ Roman"},
  \texttt{"Arial"}, etc. Enquanto em sistemas Linux e MacOS estarão
  \texttt{"serif"}, \texttt{"mono"}, etc.
\item
  \textbf{size} é para definir se o tamanho da fonte.
\end{itemize}
\end{frame}

\begin{frame}[fragile]
\begin{Shaded}
\begin{Highlighting}[]
\NormalTok{iris}\SpecialCharTok{\%\textgreater{}\%}\FunctionTok{ggplot}\NormalTok{(}\FunctionTok{aes}\NormalTok{(}\AttributeTok{x=}\NormalTok{Species, }\AttributeTok{y=}\NormalTok{Petal.Length, }\AttributeTok{fill=}\NormalTok{Species))}\SpecialCharTok{+}
  \FunctionTok{geom\_boxplot}\NormalTok{()}\SpecialCharTok{+}
  \FunctionTok{labs}\NormalTok{(}\AttributeTok{y=}\StringTok{"Comprimento de pétala"}\NormalTok{, }\AttributeTok{x=}\StringTok{"Espécies"}\NormalTok{)}\SpecialCharTok{+}
  \FunctionTok{theme}\NormalTok{(}\AttributeTok{text =} \FunctionTok{element\_text}\NormalTok{(}\AttributeTok{face=}\StringTok{"bold"}\NormalTok{,}
                            \AttributeTok{family=}\StringTok{"serif"}\NormalTok{))}
\end{Highlighting}
\end{Shaded}

\pandocbounded{\includegraphics[keepaspectratio]{index_beamer_files/figure-beamer/unnamed-chunk-27-1.pdf}}
\end{frame}

\begin{frame}[fragile]
\begin{Shaded}
\begin{Highlighting}[]
\NormalTok{iris}\SpecialCharTok{\%\textgreater{}\%}\FunctionTok{ggplot}\NormalTok{(}\FunctionTok{aes}\NormalTok{(}\AttributeTok{x=}\NormalTok{Species, }\AttributeTok{y=}\NormalTok{Petal.Length, }\AttributeTok{fill=}\NormalTok{Species))}\SpecialCharTok{+}
  \FunctionTok{geom\_boxplot}\NormalTok{()}\SpecialCharTok{+}
  \FunctionTok{labs}\NormalTok{(}\AttributeTok{y=}\StringTok{"Comprimento de pétala"}\NormalTok{, }\AttributeTok{x=}\StringTok{"Espécies"}\NormalTok{)}\SpecialCharTok{+}
  \FunctionTok{theme}\NormalTok{(}\AttributeTok{text =} \FunctionTok{element\_text}\NormalTok{(}\AttributeTok{face =} \StringTok{"bold.italic"}\NormalTok{,}
                            \AttributeTok{family=}\StringTok{"mono"}\NormalTok{, }\AttributeTok{size=}\DecValTok{16}\NormalTok{))}
\end{Highlighting}
\end{Shaded}

\pandocbounded{\includegraphics[keepaspectratio]{index_beamer_files/figure-beamer/unnamed-chunk-28-1.pdf}}
\end{frame}

\begin{frame}[fragile]
\begin{Shaded}
\begin{Highlighting}[]
\NormalTok{iris}\SpecialCharTok{\%\textgreater{}\%}\FunctionTok{ggplot}\NormalTok{(}\FunctionTok{aes}\NormalTok{(}\AttributeTok{x=}\NormalTok{Species, }\AttributeTok{y=}\NormalTok{Petal.Length, }\AttributeTok{fill=}\NormalTok{Species))}\SpecialCharTok{+}
  \FunctionTok{geom\_boxplot}\NormalTok{()}\SpecialCharTok{+}
  \FunctionTok{labs}\NormalTok{(}\AttributeTok{y=}\StringTok{"Comprimento de pétala"}\NormalTok{, }\AttributeTok{x=}\StringTok{"Espécies"}\NormalTok{)}\SpecialCharTok{+}
  \FunctionTok{theme}\NormalTok{(}\AttributeTok{text =} \FunctionTok{element\_text}\NormalTok{(}\AttributeTok{face=}\StringTok{"italic"}\NormalTok{))}
\end{Highlighting}
\end{Shaded}

\pandocbounded{\includegraphics[keepaspectratio]{index_beamer_files/figure-beamer/unnamed-chunk-29-1.pdf}}
\end{frame}

\begin{frame}[fragile]
\begin{Shaded}
\begin{Highlighting}[]
\NormalTok{iris}\SpecialCharTok{\%\textgreater{}\%}\FunctionTok{ggplot}\NormalTok{(}\FunctionTok{aes}\NormalTok{(}\AttributeTok{x=}\NormalTok{Species, }\AttributeTok{y=}\NormalTok{Petal.Length, }\AttributeTok{fill=}\NormalTok{Species))}\SpecialCharTok{+}
  \FunctionTok{geom\_boxplot}\NormalTok{()}\SpecialCharTok{+}
  \FunctionTok{labs}\NormalTok{(}\AttributeTok{y=}\StringTok{"Comprimento de pétala"}\NormalTok{, }\AttributeTok{x=}\StringTok{"Espécies"}\NormalTok{)}\SpecialCharTok{+}
  \FunctionTok{theme}\NormalTok{(}\AttributeTok{axis.text.x =} \FunctionTok{element\_text}\NormalTok{(}\AttributeTok{face=}\StringTok{"italic"}\NormalTok{))}
\end{Highlighting}
\end{Shaded}

\pandocbounded{\includegraphics[keepaspectratio]{index_beamer_files/figure-beamer/unnamed-chunk-30-1.pdf}}
\end{frame}

\begin{frame}[fragile]
\begin{Shaded}
\begin{Highlighting}[]
\NormalTok{iris}\SpecialCharTok{\%\textgreater{}\%}\FunctionTok{ggplot}\NormalTok{(}\FunctionTok{aes}\NormalTok{(}\AttributeTok{x=}\NormalTok{Species, }\AttributeTok{y=}\NormalTok{Petal.Length, }\AttributeTok{fill=}\NormalTok{Species))}\SpecialCharTok{+}
  \FunctionTok{geom\_boxplot}\NormalTok{()}\SpecialCharTok{+}
  \FunctionTok{labs}\NormalTok{(}\AttributeTok{y=}\StringTok{"Comprimento de pétala"}\NormalTok{, }\AttributeTok{x=}\StringTok{"Espécies"}\NormalTok{, }\AttributeTok{fill=}\StringTok{"Espécies"}\NormalTok{,}
       \AttributeTok{title=}\StringTok{"Aqui é o título"}\NormalTok{)}\SpecialCharTok{+}
  \FunctionTok{theme}\NormalTok{(}\AttributeTok{axis.text.x =} \FunctionTok{element\_text}\NormalTok{(}\AttributeTok{face=}\StringTok{"italic"}\NormalTok{), }
        \AttributeTok{plot.title =} \FunctionTok{element\_text}\NormalTok{(}\AttributeTok{face=}\StringTok{"bold"}\NormalTok{))}
\end{Highlighting}
\end{Shaded}

\pandocbounded{\includegraphics[keepaspectratio]{index_beamer_files/figure-beamer/unnamed-chunk-31-1.pdf}}
\end{frame}

\begin{frame}[fragile]{Manipulação da legenda}
\phantomsection\label{manipulauxe7uxe3o-da-legenda}
Caso queremos tirar a legenda ou alterar a posição da legenda,
utilizaremos o argumento \texttt{legend.position\ =}:

\begin{itemize}
\item
  \textbf{``none''} para tirar a legenda
\item
  \textbf{``top''} para a legenda ficar em cima
\item
  \textbf{``bottom''} para a legenda ficar em baixo
\item
  \textbf{``left''} para a legenda ficar na esquerda
\item
  \textbf{``right''} para a legenda ficar na direita
\end{itemize}
\end{frame}

\begin{frame}[fragile]
\begin{Shaded}
\begin{Highlighting}[]
\NormalTok{iris}\SpecialCharTok{\%\textgreater{}\%}\FunctionTok{ggplot}\NormalTok{(}\FunctionTok{aes}\NormalTok{(}\AttributeTok{x=}\NormalTok{Species, }\AttributeTok{y=}\NormalTok{Petal.Length, }\AttributeTok{fill=}\NormalTok{Species))}\SpecialCharTok{+}
  \FunctionTok{geom\_boxplot}\NormalTok{()}\SpecialCharTok{+}
  \FunctionTok{labs}\NormalTok{(}\AttributeTok{y=}\StringTok{"Comprimento de pétala"}\NormalTok{, }\AttributeTok{x=}\StringTok{"Espécies"}\NormalTok{, }\AttributeTok{fill=}\StringTok{"Espécies"}\NormalTok{,}
       \AttributeTok{title=}\StringTok{"Aqui é o título"}\NormalTok{)}\SpecialCharTok{+}
  \FunctionTok{theme}\NormalTok{(}\AttributeTok{axis.text.x =} \FunctionTok{element\_text}\NormalTok{(}\AttributeTok{face=}\StringTok{"italic"}\NormalTok{),}
        \AttributeTok{plot.title =} \FunctionTok{element\_text}\NormalTok{(}\AttributeTok{face=}\StringTok{"bold"}\NormalTok{),}
        \AttributeTok{legend.position =} \StringTok{"none"}\NormalTok{)}
\end{Highlighting}
\end{Shaded}

\pandocbounded{\includegraphics[keepaspectratio]{index_beamer_files/figure-beamer/unnamed-chunk-32-1.pdf}}
\end{frame}

\begin{frame}[fragile]
\begin{Shaded}
\begin{Highlighting}[]
\NormalTok{iris}\SpecialCharTok{\%\textgreater{}\%}\FunctionTok{ggplot}\NormalTok{(}\FunctionTok{aes}\NormalTok{(}\AttributeTok{x=}\NormalTok{Species, }\AttributeTok{y=}\NormalTok{Petal.Length, }\AttributeTok{fill=}\NormalTok{Species))}\SpecialCharTok{+}
  \FunctionTok{geom\_boxplot}\NormalTok{()}\SpecialCharTok{+}
  \FunctionTok{labs}\NormalTok{(}\AttributeTok{y=}\StringTok{"Comprimento de pétala"}\NormalTok{, }\AttributeTok{x=}\StringTok{"Espécies"}\NormalTok{, }\AttributeTok{fill=}\StringTok{"Espécies"}\NormalTok{,}
       \AttributeTok{title=}\StringTok{"Aqui é o título"}\NormalTok{)}\SpecialCharTok{+}
  \FunctionTok{theme}\NormalTok{(}\AttributeTok{axis.text.x =} \FunctionTok{element\_text}\NormalTok{(}\AttributeTok{face=}\StringTok{"italic"}\NormalTok{),}
        \AttributeTok{plot.title =} \FunctionTok{element\_text}\NormalTok{(}\AttributeTok{face=}\StringTok{"bold"}\NormalTok{),}
        \AttributeTok{legend.position =} \StringTok{"top"}\NormalTok{)}
\end{Highlighting}
\end{Shaded}

\pandocbounded{\includegraphics[keepaspectratio]{index_beamer_files/figure-beamer/unnamed-chunk-33-1.pdf}}
\end{frame}

\begin{frame}[fragile]
\begin{Shaded}
\begin{Highlighting}[]
\NormalTok{iris}\SpecialCharTok{\%\textgreater{}\%}\FunctionTok{ggplot}\NormalTok{(}\FunctionTok{aes}\NormalTok{(}\AttributeTok{x=}\NormalTok{Species, }\AttributeTok{y=}\NormalTok{Petal.Length, }\AttributeTok{fill=}\NormalTok{Species))}\SpecialCharTok{+}
  \FunctionTok{geom\_boxplot}\NormalTok{()}\SpecialCharTok{+}
  \FunctionTok{labs}\NormalTok{(}\AttributeTok{y=}\StringTok{"Comprimento de pétala"}\NormalTok{, }\AttributeTok{x=}\StringTok{"Espécies"}\NormalTok{, }\AttributeTok{fill=}\StringTok{"Espécies"}\NormalTok{,}
       \AttributeTok{title=}\StringTok{"Aqui é o título"}\NormalTok{)}\SpecialCharTok{+}
  \FunctionTok{theme}\NormalTok{(}\AttributeTok{axis.text.x =} \FunctionTok{element\_text}\NormalTok{(}\AttributeTok{face=}\StringTok{"italic"}\NormalTok{),}
        \AttributeTok{plot.title =} \FunctionTok{element\_text}\NormalTok{(}\AttributeTok{face=}\StringTok{"bold"}\NormalTok{),}
        \AttributeTok{legend.position =} \StringTok{"bottom"}\NormalTok{)}
\end{Highlighting}
\end{Shaded}

\pandocbounded{\includegraphics[keepaspectratio]{index_beamer_files/figure-beamer/unnamed-chunk-34-1.pdf}}
\end{frame}

\begin{frame}[fragile]
\begin{Shaded}
\begin{Highlighting}[]
\NormalTok{iris}\SpecialCharTok{\%\textgreater{}\%}\FunctionTok{ggplot}\NormalTok{(}\FunctionTok{aes}\NormalTok{(}\AttributeTok{x=}\NormalTok{Species, }\AttributeTok{y=}\NormalTok{Petal.Length, }\AttributeTok{fill=}\NormalTok{Species))}\SpecialCharTok{+}
  \FunctionTok{geom\_boxplot}\NormalTok{()}\SpecialCharTok{+}
  \FunctionTok{labs}\NormalTok{(}\AttributeTok{y=}\StringTok{"Comprimento de pétala"}\NormalTok{, }\AttributeTok{x=}\StringTok{"Espécies"}\NormalTok{, }\AttributeTok{fill=}\StringTok{"Espécies"}\NormalTok{,}
       \AttributeTok{title=}\StringTok{"Aqui é o título"}\NormalTok{)}\SpecialCharTok{+}
  \FunctionTok{theme}\NormalTok{(}\AttributeTok{axis.text.x =} \FunctionTok{element\_text}\NormalTok{(}\AttributeTok{face=}\StringTok{"italic"}\NormalTok{),}
        \AttributeTok{plot.title =} \FunctionTok{element\_text}\NormalTok{(}\AttributeTok{face=}\StringTok{"bold"}\NormalTok{),}
        \AttributeTok{legend.position =} \StringTok{"left"}\NormalTok{,}
        \AttributeTok{legend.text =} \FunctionTok{element\_text}\NormalTok{(}\AttributeTok{face=}\StringTok{"italic"}\NormalTok{))}
\end{Highlighting}
\end{Shaded}

\pandocbounded{\includegraphics[keepaspectratio]{index_beamer_files/figure-beamer/unnamed-chunk-35-1.pdf}}
\end{frame}

\begin{frame}[fragile]{Anotação em gráfico}
\phantomsection\label{anotauxe7uxe3o-em-gruxe1fico}
\begin{Shaded}
\begin{Highlighting}[]
\NormalTok{iris}\SpecialCharTok{\%\textgreater{}\%}\FunctionTok{ggplot}\NormalTok{(}\FunctionTok{aes}\NormalTok{(}\AttributeTok{x=}\NormalTok{Species, }\AttributeTok{y=}\NormalTok{Petal.Length, }\AttributeTok{fill=}\NormalTok{Species))}\SpecialCharTok{+}
  \FunctionTok{geom\_boxplot}\NormalTok{()}\SpecialCharTok{+}
  \FunctionTok{labs}\NormalTok{(}\AttributeTok{y=}\StringTok{"Comprimento de pétala"}\NormalTok{, }\AttributeTok{x=}\StringTok{"Espécies"}\NormalTok{, }\AttributeTok{fill=}\StringTok{"Espécies"}\NormalTok{,}
       \AttributeTok{title=}\StringTok{"Aqui é o título"}\NormalTok{)}\SpecialCharTok{+}
  \FunctionTok{theme}\NormalTok{(}\AttributeTok{axis.text.x =} \FunctionTok{element\_text}\NormalTok{(}\AttributeTok{face=}\StringTok{"italic"}\NormalTok{),}
        \AttributeTok{plot.title =} \FunctionTok{element\_text}\NormalTok{(}\AttributeTok{face=}\StringTok{"bold"}\NormalTok{),}
        \AttributeTok{legend.position =} \StringTok{"left"}\NormalTok{,}
        \AttributeTok{legend.text =} \FunctionTok{element\_text}\NormalTok{(}\AttributeTok{face=}\StringTok{"italic"}\NormalTok{))}\SpecialCharTok{+}
  \FunctionTok{geom\_text}\NormalTok{(}\AttributeTok{x =} \FloatTok{2.5}\NormalTok{, }\AttributeTok{y =} \DecValTok{4}\NormalTok{, }\AttributeTok{label =} \StringTok{"Ponto importante"}\NormalTok{,}
            \AttributeTok{color =} \StringTok{"red"}\NormalTok{, }\AttributeTok{face=}\StringTok{"bold"}\NormalTok{)}
\end{Highlighting}
\end{Shaded}
\end{frame}

\begin{frame}
\pandocbounded{\includegraphics[keepaspectratio]{index_beamer_files/figure-beamer/unnamed-chunk-37-1.pdf}}
\end{frame}

\begin{frame}[fragile]
\begin{Shaded}
\begin{Highlighting}[]
\NormalTok{iris}\SpecialCharTok{\%\textgreater{}\%}\FunctionTok{ggplot}\NormalTok{(}\FunctionTok{aes}\NormalTok{(}\AttributeTok{x=}\NormalTok{Species, }\AttributeTok{y=}\NormalTok{Petal.Length, }\AttributeTok{fill=}\NormalTok{Species))}\SpecialCharTok{+}
  \FunctionTok{geom\_boxplot}\NormalTok{()}\SpecialCharTok{+}
  \FunctionTok{labs}\NormalTok{(}\AttributeTok{y=}\StringTok{"Comprimento de pétala"}\NormalTok{, }\AttributeTok{x=}\StringTok{"Espécies"}\NormalTok{, }\AttributeTok{fill=}\StringTok{"Espécies"}\NormalTok{,}
       \AttributeTok{title=}\StringTok{"Aqui é o título"}\NormalTok{)}\SpecialCharTok{+}
  \FunctionTok{theme}\NormalTok{(}\AttributeTok{axis.text.x =} \FunctionTok{element\_text}\NormalTok{(}\AttributeTok{face=}\StringTok{"italic"}\NormalTok{), }
        \AttributeTok{plot.title =} \FunctionTok{element\_text}\NormalTok{(}\AttributeTok{face=}\StringTok{"bold"}\NormalTok{), }
        \AttributeTok{legend.position =} \StringTok{"left"}\NormalTok{, }
        \AttributeTok{legend.text =} \FunctionTok{element\_text}\NormalTok{(}\AttributeTok{face=}\StringTok{"italic"}\NormalTok{))}\SpecialCharTok{+}
  \FunctionTok{geom\_text}\NormalTok{(}\AttributeTok{x =} \FloatTok{2.5}\NormalTok{, }\AttributeTok{y =} \DecValTok{4}\NormalTok{, }\AttributeTok{label =} \StringTok{"Ponto importante"}\NormalTok{,}
            \AttributeTok{color =} \StringTok{"red"}\NormalTok{)}\SpecialCharTok{+}
  \FunctionTok{annotate}\NormalTok{(}\StringTok{"vline"}\NormalTok{, }\AttributeTok{x =} \DecValTok{2}\NormalTok{, }\AttributeTok{xintercept =} \DecValTok{2}\NormalTok{, }\AttributeTok{linetype =} \StringTok{"dashed"}\NormalTok{,}
           \AttributeTok{color =} \StringTok{"blue"}\NormalTok{)}
\end{Highlighting}
\end{Shaded}
\end{frame}

\begin{frame}
\pandocbounded{\includegraphics[keepaspectratio]{index_beamer_files/figure-beamer/unnamed-chunk-39-1.pdf}}
\end{frame}

\begin{frame}[fragile]
\begin{Shaded}
\begin{Highlighting}[]
\NormalTok{iris}\SpecialCharTok{\%\textgreater{}\%}\FunctionTok{ggplot}\NormalTok{(}\FunctionTok{aes}\NormalTok{(}\AttributeTok{x=}\NormalTok{Species, }\AttributeTok{y=}\NormalTok{Petal.Length, }\AttributeTok{fill=}\NormalTok{Species))}\SpecialCharTok{+}
  \FunctionTok{geom\_boxplot}\NormalTok{()}\SpecialCharTok{+}
  \FunctionTok{labs}\NormalTok{(}\AttributeTok{y=}\StringTok{"Comprimento de pétala"}\NormalTok{, }\AttributeTok{x=}\StringTok{"Espécies"}\NormalTok{, }\AttributeTok{fill=}\StringTok{"Espécies"}\NormalTok{,}
       \AttributeTok{title=}\StringTok{"Aqui é o título"}\NormalTok{)}\SpecialCharTok{+}
  \FunctionTok{theme}\NormalTok{(}\AttributeTok{axis.text.x =} \FunctionTok{element\_text}\NormalTok{(}\AttributeTok{face=}\StringTok{"italic"}\NormalTok{),}
        \AttributeTok{plot.title =} \FunctionTok{element\_text}\NormalTok{(}\AttributeTok{face=}\StringTok{"bold"}\NormalTok{),}
        \AttributeTok{legend.position =} \StringTok{"left"}\NormalTok{,}
        \AttributeTok{legend.text =} \FunctionTok{element\_text}\NormalTok{(}\AttributeTok{face=}\StringTok{"italic"}\NormalTok{))}\SpecialCharTok{+}
  \FunctionTok{geom\_text}\NormalTok{(}\AttributeTok{x =} \FloatTok{2.5}\NormalTok{, }\AttributeTok{y =} \DecValTok{4}\NormalTok{, }\AttributeTok{label =} \StringTok{"Ponto importante"}\NormalTok{,}
            \AttributeTok{color =} \StringTok{"red"}\NormalTok{)}\SpecialCharTok{+}
  \FunctionTok{annotate}\NormalTok{(}\StringTok{"text"}\NormalTok{, }\AttributeTok{x =} \DecValTok{1}\NormalTok{, }\AttributeTok{y =} \FloatTok{3.5}\NormalTok{, }\AttributeTok{label =} \StringTok{"outro ponto"}\NormalTok{,}
           \AttributeTok{color =} \StringTok{"blue"}\NormalTok{)}
\end{Highlighting}
\end{Shaded}
\end{frame}

\begin{frame}
\pandocbounded{\includegraphics[keepaspectratio]{index_beamer_files/figure-beamer/unnamed-chunk-41-1.pdf}}
\end{frame}

\begin{frame}{Temas (\texttt{theme\_*})}
\phantomsection\label{temas-theme_}
\pandocbounded{\includegraphics[keepaspectratio]{index_beamer_files/figure-beamer/Box-plot tema-1.pdf}}
\end{frame}

\begin{frame}[fragile]{Unindo vários gráficos em uma imagem só}
\phantomsection\label{unindo-vuxe1rios-gruxe1ficos-em-uma-imagem-suxf3}
\begin{Shaded}
\begin{Highlighting}[]
\CommentTok{\#Criando ggplots}
\NormalTok{barra}\OtherTok{\textless{}{-}}\NormalTok{Escolaridade}\SpecialCharTok{\%\textgreater{}\%}
  \FunctionTok{mutate}\NormalTok{(}\AttributeTok{Escolaridade=}\FunctionTok{fct\_relevel}\NormalTok{(Escolaridade,}\StringTok{"Fundamental"}\NormalTok{,}\StringTok{"Médio"}\NormalTok{, }\StringTok{"Graduação"}\NormalTok{))}\SpecialCharTok{\%\textgreater{}\%}
  \FunctionTok{ggplot}\NormalTok{(}\FunctionTok{aes}\NormalTok{(}\AttributeTok{x=}\NormalTok{Escolaridade))}\SpecialCharTok{+}
  \FunctionTok{geom\_bar}\NormalTok{()}\SpecialCharTok{+}
  \FunctionTok{labs}\NormalTok{(}\AttributeTok{y=}\StringTok{"Frequência"}\NormalTok{, }\AttributeTok{x=}\StringTok{"Escolaridade"}\NormalTok{)}

\NormalTok{polígono}\OtherTok{\textless{}{-}}\NormalTok{iris}\SpecialCharTok{\%\textgreater{}\%}
  \FunctionTok{ggplot}\NormalTok{(}\FunctionTok{aes}\NormalTok{(}\AttributeTok{x=}\NormalTok{Sepal.Length))}\SpecialCharTok{+}
  \FunctionTok{geom\_freqpoly}\NormalTok{()}\SpecialCharTok{+}
  \FunctionTok{labs}\NormalTok{(}\AttributeTok{y=}\StringTok{"Frequência"}\NormalTok{,}\AttributeTok{x=}\StringTok{"Comprimento de Sépala"}\NormalTok{)}

\NormalTok{boxplot}\OtherTok{\textless{}{-}}\NormalTok{iris}\SpecialCharTok{\%\textgreater{}\%}
  \FunctionTok{ggplot}\NormalTok{(}\FunctionTok{aes}\NormalTok{(}\AttributeTok{y=}\NormalTok{Sepal.Length, }\AttributeTok{x=}\NormalTok{Species))}\SpecialCharTok{+}
  \FunctionTok{geom\_boxplot}\NormalTok{()}\SpecialCharTok{+}
  \FunctionTok{labs}\NormalTok{(}\AttributeTok{y=}\StringTok{"Comprimento de Sépala"}\NormalTok{, }\AttributeTok{x=}\StringTok{"Espécies"}\NormalTok{)}
\end{Highlighting}
\end{Shaded}
\end{frame}

\begin{frame}[fragile]
\begin{Shaded}
\begin{Highlighting}[]
\NormalTok{pontos}\OtherTok{\textless{}{-}}\NormalTok{iris}\SpecialCharTok{\%\textgreater{}\%}
  \FunctionTok{ggplot}\NormalTok{(}\FunctionTok{aes}\NormalTok{(}\AttributeTok{x=}\NormalTok{Sepal.Length,}\AttributeTok{y=}\NormalTok{Sepal.Width, }\AttributeTok{color=}\NormalTok{Species))}\SpecialCharTok{+}
  \FunctionTok{geom\_point}\NormalTok{()}\SpecialCharTok{+}
  \FunctionTok{labs}\NormalTok{(}\AttributeTok{x=}\StringTok{"Comprimento de Sépala"}\NormalTok{, }\AttributeTok{y=}\StringTok{"Largura de Sépala"}\NormalTok{, }\AttributeTok{color=}\StringTok{"Espécies"}\NormalTok{)}
\end{Highlighting}
\end{Shaded}
\end{frame}

\begin{frame}[fragile]
\begin{Shaded}
\begin{Highlighting}[]
\NormalTok{barra }\SpecialCharTok{+}\NormalTok{ polígono }\SpecialCharTok{+}\NormalTok{ boxplot }\SpecialCharTok{+}\NormalTok{ pontos}
\end{Highlighting}
\end{Shaded}

\pandocbounded{\includegraphics[keepaspectratio]{index_beamer_files/figure-beamer/unnamed-chunk-44-1.pdf}}
\end{frame}

\begin{frame}[fragile]
\begin{enumerate}
\setcounter{enumi}{2}
\tightlist
\item
  Também é possível utilizar diferêntes conformações utilizando
  elementos matemáticos, como \texttt{/} e \texttt{()}.
\end{enumerate}

\pandocbounded{\includegraphics[keepaspectratio]{index_beamer_files/figure-beamer/unnamed-chunk-45-1.pdf}}
\end{frame}

\section{Extra}\label{extra}

\begin{frame}[fragile]{Mapas}
\phantomsection\label{mapas}
\begin{Shaded}
\begin{Highlighting}[]
\CommentTok{\#instalando o pacote raster e sf}
\FunctionTok{install.packages}\NormalTok{(}\StringTok{"raster"}\NormalTok{)}
\FunctionTok{install.packages}\NormalTok{(}\StringTok{"sf"}\NormalTok{)}

\CommentTok{\#carregando o pacote raster e sf}
\FunctionTok{library}\NormalTok{(raster)}
\FunctionTok{library}\NormalTok{(sf)}
\end{Highlighting}
\end{Shaded}
\end{frame}

\begin{frame}[fragile]
\begin{Shaded}
\begin{Highlighting}[]
\CommentTok{\# Importando dados}
\NormalTok{prec}\OtherTok{\textless{}{-}}\FunctionTok{raster}\NormalTok{(}\StringTok{"pelprec.tiff"}\NormalTok{)}

\NormalTok{pel}\OtherTok{\textless{}{-}}\FunctionTok{read\_sf}\NormalTok{(}\StringTok{"Pelotas/Pelotas.shp"}\NormalTok{)}

\CommentTok{\# Convertendo raster para data frame para o ggplot processar o dado}
\NormalTok{prec\_df}\OtherTok{\textless{}{-}}\FunctionTok{as.data.frame}\NormalTok{(prec, }\AttributeTok{xy =} \ConstantTok{TRUE}\NormalTok{, }\AttributeTok{na.rm =} \ConstantTok{TRUE}\NormalTok{)}

\FunctionTok{head}\NormalTok{(prec\_df)}
\end{Highlighting}
\end{Shaded}

\begin{verbatim}
           x         y pelprec
14 -52.49583 -31.32917     120
15 -52.48750 -31.32917     121
16 -52.47917 -31.32917     121
17 -52.47083 -31.32917     120
18 -52.46250 -31.32917     120
19 -52.45417 -31.32917     120
\end{verbatim}
\end{frame}

\begin{frame}[fragile]
\begin{Shaded}
\begin{Highlighting}[]
\FunctionTok{ggplot}\NormalTok{(prec\_df,}\FunctionTok{aes}\NormalTok{(}\AttributeTok{x=}\NormalTok{x,}\AttributeTok{y=}\NormalTok{y,}\AttributeTok{fill=}\NormalTok{pelprec))}\SpecialCharTok{+}
  \FunctionTok{geom\_raster}\NormalTok{()}
\end{Highlighting}
\end{Shaded}

\pandocbounded{\includegraphics[keepaspectratio]{index_beamer_files/figure-beamer/unnamed-chunk-48-1.pdf}}
\end{frame}

\begin{frame}[fragile]
\begin{Shaded}
\begin{Highlighting}[]
\CommentTok{\# Cores padrão}
\FunctionTok{ggplot}\NormalTok{()}\SpecialCharTok{+}
  \FunctionTok{geom\_raster}\NormalTok{(}\AttributeTok{data=}\NormalTok{prec\_df,}\FunctionTok{aes}\NormalTok{(}\AttributeTok{x=}\NormalTok{x,}\AttributeTok{y=}\NormalTok{y,}\AttributeTok{fill=}\NormalTok{pelprec))}\SpecialCharTok{+}
  \FunctionTok{geom\_sf}\NormalTok{(}\AttributeTok{data=}\NormalTok{pel,}\AttributeTok{fill=}\ConstantTok{NA}\NormalTok{, }\AttributeTok{color=}\StringTok{"gray"}\NormalTok{,}\AttributeTok{linewidth=}\DecValTok{2}\NormalTok{, }\AttributeTok{alpha=}\NormalTok{.}\DecValTok{01}\NormalTok{)}\SpecialCharTok{+}
  \FunctionTok{labs}\NormalTok{(}\AttributeTok{title=}\StringTok{"Mapa da média anual da precipitação }\SpecialCharTok{\textbackslash{}n}
\StringTok{       em Pelotas{-}RS entre 1970{-}2000"}\NormalTok{,}
       \AttributeTok{y=}\StringTok{"Latitude"}\NormalTok{,}
       \AttributeTok{x=}\StringTok{"Longitude"}\NormalTok{,}
       \AttributeTok{fill=}\StringTok{"Precipitação (mm)"}\NormalTok{)}\SpecialCharTok{+}
  \FunctionTok{theme\_bw}\NormalTok{()}
\end{Highlighting}
\end{Shaded}
\end{frame}

\begin{frame}
\pandocbounded{\includegraphics[keepaspectratio]{index_beamer_files/figure-beamer/unnamed-chunk-50-1.pdf}}
\end{frame}

\begin{frame}[fragile]
\begin{Shaded}
\begin{Highlighting}[]
\FunctionTok{ggplot}\NormalTok{()}\SpecialCharTok{+}
  \FunctionTok{geom\_raster}\NormalTok{(}\AttributeTok{data=}\NormalTok{prec\_df,}\FunctionTok{aes}\NormalTok{(}\AttributeTok{x=}\NormalTok{x,}\AttributeTok{y=}\NormalTok{y,}\AttributeTok{fill=}\NormalTok{pelprec))}\SpecialCharTok{+}
  \FunctionTok{geom\_sf}\NormalTok{(}\AttributeTok{data=}\NormalTok{pel,}\AttributeTok{fill=}\ConstantTok{NA}\NormalTok{, }\AttributeTok{color=}\StringTok{"gray"}\NormalTok{,}\AttributeTok{linewidth=}\DecValTok{2}\NormalTok{, }\AttributeTok{alpha=}\NormalTok{.}\DecValTok{01}\NormalTok{)}\SpecialCharTok{+}
  \FunctionTok{labs}\NormalTok{(}\AttributeTok{title=}\StringTok{"Mapa da média anual da precipitação }\SpecialCharTok{\textbackslash{}n}\StringTok{ }
\StringTok{       em Pelotas{-}RS entre 1970{-}2000"}\NormalTok{,}
       \AttributeTok{y=}\StringTok{"Latitude"}\NormalTok{,}
       \AttributeTok{x=}\StringTok{"Longitude"}\NormalTok{,}
       \AttributeTok{fill=}\StringTok{"Precipitação (mm)"}\NormalTok{)}\SpecialCharTok{+}
  \FunctionTok{theme\_bw}\NormalTok{()}\SpecialCharTok{+}
  \FunctionTok{scale\_fill\_gradient}\NormalTok{(}\AttributeTok{low=}\StringTok{"gray"}\NormalTok{,}\AttributeTok{high=}\StringTok{"blue"}\NormalTok{)}
\end{Highlighting}
\end{Shaded}
\end{frame}

\begin{frame}
\pandocbounded{\includegraphics[keepaspectratio]{index_beamer_files/figure-beamer/unnamed-chunk-52-1.pdf}}
\end{frame}

\begin{frame}[fragile]
\begin{Shaded}
\begin{Highlighting}[]
\FunctionTok{ggplot}\NormalTok{()}\SpecialCharTok{+}
  \FunctionTok{geom\_raster}\NormalTok{(}\AttributeTok{data=}\NormalTok{prec\_df,}\FunctionTok{aes}\NormalTok{(}\AttributeTok{x=}\NormalTok{x,}\AttributeTok{y=}\NormalTok{y,}\AttributeTok{fill=}\NormalTok{pelprec))}\SpecialCharTok{+}
  \FunctionTok{geom\_sf}\NormalTok{(}\AttributeTok{data=}\NormalTok{pel,}\AttributeTok{fill=}\ConstantTok{NA}\NormalTok{, }\AttributeTok{color=}\StringTok{"gray"}\NormalTok{,}\AttributeTok{linewidth=}\DecValTok{2}\NormalTok{, }\AttributeTok{alpha=}\NormalTok{.}\DecValTok{01}\NormalTok{)}\SpecialCharTok{+}
  \FunctionTok{labs}\NormalTok{(}\AttributeTok{title=}\StringTok{"Mapa da média anual da precipitação }\SpecialCharTok{\textbackslash{}n}
\StringTok{       em Pelotas{-}RS entre 1970{-}2000"}\NormalTok{,}
       \AttributeTok{y=}\StringTok{"Latitude"}\NormalTok{,}
       \AttributeTok{x=}\StringTok{"Longitude"}\NormalTok{,}
       \AttributeTok{fill=}\StringTok{"Precipitação (mm)"}\NormalTok{)}\SpecialCharTok{+}
  \FunctionTok{theme\_bw}\NormalTok{()}\SpecialCharTok{+}
  \FunctionTok{scale\_fill\_gradientn}\NormalTok{(}\AttributeTok{colours =} \FunctionTok{terrain.colors}\NormalTok{(}\DecValTok{10}\NormalTok{))}
\end{Highlighting}
\end{Shaded}
\end{frame}

\begin{frame}
\pandocbounded{\includegraphics[keepaspectratio]{index_beamer_files/figure-beamer/unnamed-chunk-54-1.pdf}}
\end{frame}

\begin{frame}[fragile]{Paleta de cores para daltônicos}
\phantomsection\label{paleta-de-cores-para-daltuxf4nicos}
\begin{Shaded}
\begin{Highlighting}[]
\CommentTok{\#intalando pacote viridis}
\FunctionTok{install.packages}\NormalTok{(}\StringTok{"viridis"}\NormalTok{)}
\CommentTok{\#carregando pacote viridis}
\FunctionTok{library}\NormalTok{(viridis)}

\FunctionTok{ggplot}\NormalTok{()}\SpecialCharTok{+}
  \FunctionTok{geom\_raster}\NormalTok{(}\AttributeTok{data=}\NormalTok{prec\_df,}\FunctionTok{aes}\NormalTok{(}\AttributeTok{x=}\NormalTok{x,}\AttributeTok{y=}\NormalTok{y,}\AttributeTok{fill=}\NormalTok{pelprec))}\SpecialCharTok{+}
  \FunctionTok{geom\_sf}\NormalTok{(}\AttributeTok{data=}\NormalTok{pel,}\AttributeTok{fill=}\ConstantTok{NA}\NormalTok{, }\AttributeTok{color=}\StringTok{"gray"}\NormalTok{,}\AttributeTok{linewidth=}\DecValTok{2}\NormalTok{, }\AttributeTok{alpha=}\NormalTok{.}\DecValTok{01}\NormalTok{)}\SpecialCharTok{+}
  \FunctionTok{labs}\NormalTok{(}\AttributeTok{title=}\StringTok{"Mapa da média anual da precipitação }\SpecialCharTok{\textbackslash{}n}
\StringTok{       em Pelotas{-}RS entre 1970{-}2000"}\NormalTok{,}
       \AttributeTok{y=}\StringTok{"Latitude"}\NormalTok{,}
       \AttributeTok{x=}\StringTok{"Longitude"}\NormalTok{,}
       \AttributeTok{fill=}\StringTok{"Precipitação (mm)"}\NormalTok{)}\SpecialCharTok{+}
  \FunctionTok{theme\_bw}\NormalTok{()}\SpecialCharTok{+}
  \FunctionTok{scale\_fill\_viridis}\NormalTok{()}
\end{Highlighting}
\end{Shaded}
\end{frame}

\begin{frame}
\pandocbounded{\includegraphics[keepaspectratio]{index_beamer_files/figure-beamer/unnamed-chunk-56-1.pdf}}
\end{frame}

\begin{frame}[fragile]{Temas divertidos}
\phantomsection\label{temas-divertidos}
\begin{Shaded}
\begin{Highlighting}[]
\FunctionTok{install.packages}\NormalTok{(}\StringTok{"remotes"}\NormalTok{)}
\NormalTok{remotes}\SpecialCharTok{::}\FunctionTok{install\_github}\NormalTok{(}\StringTok{"MatthewBJane/ThemePark"}\NormalTok{)}
\FunctionTok{library}\NormalTok{(ThemePark)}

\NormalTok{iris}\SpecialCharTok{\%\textgreater{}\%}\FunctionTok{ggplot}\NormalTok{(}\FunctionTok{aes}\NormalTok{(}\AttributeTok{x=}\NormalTok{Species, }\AttributeTok{y=}\NormalTok{Petal.Length, }\AttributeTok{fill=}\NormalTok{Species))}\SpecialCharTok{+}
  \FunctionTok{geom\_boxplot}\NormalTok{(}\AttributeTok{fill=}\FunctionTok{c}\NormalTok{(lordoftherings\_theme\_colors[}\StringTok{"light"}\NormalTok{],}
\NormalTok{                      lordoftherings\_theme\_colors[}\StringTok{"medium"}\NormalTok{],}
\NormalTok{                      lordoftherings\_theme\_colors[}\StringTok{"dark"}\NormalTok{]))}\SpecialCharTok{+}
  \FunctionTok{labs}\NormalTok{(}\AttributeTok{y=}\StringTok{"Comprimento de pétala"}\NormalTok{, }\AttributeTok{x=}\StringTok{"Espécies"}\NormalTok{,}
       \AttributeTok{title=} \StringTok{"Tema Senhor dos Anéis"}\NormalTok{)}\SpecialCharTok{+}
  \FunctionTok{theme\_lordoftherings}\NormalTok{()}
\end{Highlighting}
\end{Shaded}
\end{frame}

\begin{frame}
\pandocbounded{\includegraphics[keepaspectratio]{index_beamer_files/figure-beamer/unnamed-chunk-57-1.pdf}}
\end{frame}

\begin{frame}[fragile]
\begin{Shaded}
\begin{Highlighting}[]
\NormalTok{iris}\SpecialCharTok{\%\textgreater{}\%}\FunctionTok{ggplot}\NormalTok{(}\FunctionTok{aes}\NormalTok{(}\AttributeTok{x=}\NormalTok{Species, }\AttributeTok{y=}\NormalTok{Petal.Length, }\AttributeTok{fill=}\NormalTok{Species))}\SpecialCharTok{+}
  \FunctionTok{geom\_boxplot}\NormalTok{(}\AttributeTok{fill=}\FunctionTok{c}\NormalTok{(barbie\_theme\_colors[}\StringTok{"light"}\NormalTok{],}
\NormalTok{                      barbie\_theme\_colors[}\StringTok{"medium"}\NormalTok{],}
\NormalTok{                      barbie\_theme\_colors[}\StringTok{"dark"}\NormalTok{]))}\SpecialCharTok{+}
  \FunctionTok{labs}\NormalTok{(}\AttributeTok{y=}\StringTok{"Comprimento de pétala"}\NormalTok{, }\AttributeTok{x=}\StringTok{"Espécies"}\NormalTok{,}
       \AttributeTok{title=} \StringTok{"Tema Barbie"}\NormalTok{)}\SpecialCharTok{+}
  \FunctionTok{theme\_barbie}\NormalTok{()}
\end{Highlighting}
\end{Shaded}
\end{frame}

\begin{frame}
\pandocbounded{\includegraphics[keepaspectratio]{index_beamer_files/figure-beamer/unnamed-chunk-59-1.pdf}}
\end{frame}

\begin{frame}[fragile]
\begin{Shaded}
\begin{Highlighting}[]
\NormalTok{iris}\SpecialCharTok{\%\textgreater{}\%}\FunctionTok{ggplot}\NormalTok{(}\FunctionTok{aes}\NormalTok{(}\AttributeTok{x=}\NormalTok{Species, }\AttributeTok{y=}\NormalTok{Petal.Length, }\AttributeTok{fill=}\NormalTok{Species))}\SpecialCharTok{+}
  \FunctionTok{geom\_boxplot}\NormalTok{(}\AttributeTok{fill=}\FunctionTok{c}\NormalTok{(simpsons\_theme\_colors[}\StringTok{"light"}\NormalTok{],}
\NormalTok{                      simpsons\_theme\_colors[}\StringTok{"medium"}\NormalTok{],}
\NormalTok{                      simpsons\_theme\_colors[}\StringTok{"dark"}\NormalTok{]))}\SpecialCharTok{+}
  \FunctionTok{labs}\NormalTok{(}\AttributeTok{y=}\StringTok{"Comprimento de pétala"}\NormalTok{, }\AttributeTok{x=}\StringTok{"Espécies"}\NormalTok{, }\AttributeTok{title=} \StringTok{"Tema Simpsons"}\NormalTok{)}\SpecialCharTok{+}
  \FunctionTok{theme\_simpsons}\NormalTok{()}
\end{Highlighting}
\end{Shaded}
\end{frame}

\begin{frame}
\pandocbounded{\includegraphics[keepaspectratio]{index_beamer_files/figure-beamer/unnamed-chunk-61-1.pdf}}
\end{frame}

\begin{frame}[fragile]
\begin{Shaded}
\begin{Highlighting}[]
\NormalTok{iris}\SpecialCharTok{\%\textgreater{}\%}\FunctionTok{ggplot}\NormalTok{(}\FunctionTok{aes}\NormalTok{(}\AttributeTok{x=}\NormalTok{Species, }\AttributeTok{y=}\NormalTok{Petal.Length, }\AttributeTok{fill=}\NormalTok{Species))}\SpecialCharTok{+}
  \FunctionTok{geom\_boxplot}\NormalTok{(}\AttributeTok{fill=}\FunctionTok{c}\NormalTok{(friends\_theme\_colors[}\StringTok{"light"}\NormalTok{],}
\NormalTok{                      friends\_theme\_colors[}\StringTok{"medium"}\NormalTok{],}
\NormalTok{                      friends\_theme\_colors[}\StringTok{"dark"}\NormalTok{]))}\SpecialCharTok{+}
  \FunctionTok{labs}\NormalTok{(}\AttributeTok{y=}\StringTok{"Comprimento de pétala"}\NormalTok{, }\AttributeTok{x=}\StringTok{"Espécies"}\NormalTok{, }\AttributeTok{title=} \StringTok{"Tema Friends"}\NormalTok{)}\SpecialCharTok{+}
  \FunctionTok{theme\_friends}\NormalTok{()}
\end{Highlighting}
\end{Shaded}
\end{frame}

\begin{frame}
\pandocbounded{\includegraphics[keepaspectratio]{index_beamer_files/figure-beamer/unnamed-chunk-63-1.pdf}}
\end{frame}

\begin{frame}[fragile]
\begin{Shaded}
\begin{Highlighting}[]
\NormalTok{iris}\SpecialCharTok{\%\textgreater{}\%}\FunctionTok{ggplot}\NormalTok{(}\FunctionTok{aes}\NormalTok{(}\AttributeTok{x=}\NormalTok{Species, }\AttributeTok{y=}\NormalTok{Petal.Length, }\AttributeTok{fill=}\NormalTok{Species))}\SpecialCharTok{+}
  \FunctionTok{geom\_boxplot}\NormalTok{(}\AttributeTok{fill=}\FunctionTok{c}\NormalTok{(starwars\_theme\_colors[}\StringTok{"light"}\NormalTok{],}
\NormalTok{                      starwars\_theme\_colors[}\StringTok{"medium"}\NormalTok{],}
\NormalTok{                      starwars\_theme\_colors[}\StringTok{"dark"}\NormalTok{]))}\SpecialCharTok{+}
  \FunctionTok{labs}\NormalTok{(}\AttributeTok{y=}\StringTok{"Comprimento de pétala"}\NormalTok{, }\AttributeTok{x=}\StringTok{"Espécies"}\NormalTok{,}
       \AttributeTok{title=} \StringTok{"Tema Star wars"}\NormalTok{)}\SpecialCharTok{+}
  \FunctionTok{theme\_starwars}\NormalTok{()}
\end{Highlighting}
\end{Shaded}
\end{frame}

\begin{frame}
\pandocbounded{\includegraphics[keepaspectratio]{index_beamer_files/figure-beamer/unnamed-chunk-65-1.pdf}}
\end{frame}

\begin{frame}{Referências}
\phantomsection\label{referuxeancias}
\phantomsection\label{refs}
\begin{CSLReferences}{1}{0}
\bibitem[\citeproctext]{ref-plotly}
Sievert, Carson. 2020. \emph{Interactive Web-Based Data Visualization
with r, Plotly, and Shiny}. Chapman; Hall/CRC.
\url{https://plotly-r.com}.

\bibitem[\citeproctext]{ref-ggplot2}
Wickham, Hadley. 2016. \emph{Ggplot2: Elegant Graphics for Data
Analysis}. Springer-Verlag New York.
\url{https://ggplot2.tidyverse.org}.

\bibitem[\citeproctext]{ref-forcats}
---------. 2023. \emph{Forcats: Tools for Working with Categorical
Variables (Factors)}. \url{https://forcats.tidyverse.org/}.

\bibitem[\citeproctext]{ref-dplyr}
Wickham, Hadley, Romain François, Lionel Henry, Kirill Müller, and Davis
Vaughan. 2023. \emph{Dplyr: A Grammar of Data Manipulation}.
\url{https://dplyr.tidyverse.org}.

\bibitem[\citeproctext]{ref-wilkinson2011grammar}
Wilkinson, Leland. 2011. {``The Grammar of Graphics.''} In
\emph{Handbook of Computational Statistics: Concepts and Methods},
375--414. Springer.

\end{CSLReferences}
\end{frame}

\section{Agrdeço atenção!}\label{agrdeuxe7o-atenuxe7uxe3o}

\begin{frame}{Agrdeço atenção!}
\textbf{Para mais informções}
\url{https://ggplot2.tidyverse.org/reference/index.html}

\textbf{Documentação desta oficina com maior detalhe}
\url{https://izzyreal18.github.io/oficinaggplotufpel.github.io/}
\end{frame}




\end{document}
