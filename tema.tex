%%%%%%%%%%%%%%%%%%%%%%%%%%%%%%%%%%%%%%%%%%%%%%%%%%%%%%%%%
% TEMPLATE BEAMER BASEADO NO PDF DA OFICINA EXPERIMENTAL
% CRIADO POR GEMINI (VERSÃO CORRIGIDA)
%%%%%%%%%%%%%%%%%%%%%%%%%%%%%%%%%%%%%%%%%%%%%%%%%%%%%%%%%

%\documentclass{beamer}

% --- PACOTES BÁSICOS ---
\usepackage[utf8]{inputenc}
\usepackage[T1]{fontenc}
\usepackage{graphicx}
\usepackage{lmodern}
\usepackage{xcolor}
\usepackage[absolute,overlay]{textpos} % para posicionamento do logo

% -------------------------------------
% Pacotes para código
\usepackage{listings}


% --- TEMA E CORES ---
% Usamos um tema base e personalizamos as cores e templates.
\usetheme{default}

% --- CORES INSTITUCIONAIS ---
\definecolor{UFPELBlue}{RGB}{20, 49, 129}

% --- CONFIGURAÇÃO DE CORES DO BEAMER ---
\setbeamercolor{palette primary}{bg=UFPELBlue, fg=white}
\setbeamercolor{frametitle}{bg=UFPELBlue, fg=white}
\setbeamercolor{title}{fg=UFPELBlue}
\setbeamercolor{structure}{fg=UFPELBlue}

% --- FONTES ---
\usefonttheme{default}
\setbeamerfont{title}{size=\large, series=\bfseries}
\setbeamerfont{frametitle}{size=\large, series=\bfseries}
\setbeamerfont{author}{size=\small}
\setbeamerfont{institute}{size=\small}

% --- PACOTES PARA POSICIONAMENTO DO LOGO E BARRA ---
\usepackage[absolute,overlay]{textpos}
\usepackage{tikz}

% Barra azul como footline
\setbeamertemplate{footline}{%
    \leavevmode%
    \hbox{%
        \begin{beamercolorbox}[wd=\paperwidth,ht=2.5ex,dp=1.5ex]{frametitle}
        \end{beamercolorbox}%
    }%
}

% --- TEMPLATE DO TÍTULO DO FRAME ---
\defbeamertemplate*{frametitle}{custom-frametitle}{%
  % Barra azul no topo
  \nointerlineskip
  \begin{beamercolorbox}[wd=\paperwidth, ht=2.5ex, dp=1ex]{frametitle}
  \end{beamercolorbox}

  % Título centralizado abaixo da barra azul
  \vspace{0.2cm}
  \begin{center}
    {\usebeamercolor[fg]{title}\usebeamerfont{frametitle}\insertframetitle}
  \end{center}
  \vspace{0.2cm}

  % Logo do R no canto superior direito, fixo
  \begin{textblock*}{2cm}(0.88\paperwidth,0.1cm)
    \includegraphics[height=1cm]{logo_r.png} % Substitua pelo caminho correto
  \end{textblock*}
}